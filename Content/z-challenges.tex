\chapter{\iflanguage{ngerman}{Herausforderungen}{Challenges}}
\label{sec:overview}

\subsection{Kosten- Nutzen Verhältnis}
1. Der digitale Operationssaal- Kapitel 1
	"Diese Fortschritte bringen aber auch Nachteile und neue Probleme, insbesondere
	wegen der eskalierenden Kosten im Gesundheitswesen, mit sich. Es ist auch nicht
	immer ohne Weiteres möglich, eine Kosteneffizienz dieser neuen Technologien und
	Systeme, neuer (minimal) interventioneller Verfahren sowie die Neugestaltung von
	Versorgungsinfrastrukturen, wie z. B. OP-Säle, nachzuweisen. Die Entwicklung und
	Verbreitung dieser Technologien sind zu zentralen Fragen in der Debatte über die
	Gesundheitsreform und die Finanzierung des Gesundheitswesens geworden."
	
	"Im ersten Jahrzehnt des 21. Jahrhunderts ist deutlich geworden, dass die Eskalation
	der Kosten im Gesundheitswesen nicht mehr tragbar ist, dass die Technik selbst
	ein wichtiger Treiber der steigenden Kosten im Gesundheitswesen ist und dass Ärzte
	und Patienten neue medizinische Technologien begrüßen, bevor sie sich als effektiver
	oder weniger kostspielig erwiesen haben. Relativ wenig strenge Nachweise liegen
	darüber vor, welche Behandlungen am effektivsten sind, für welchen Patienten, oder
	ob die Vorteile der teureren Therapien ihre Mehrkosten rechtfertigen [6]."
	
	"Eine Reihe anderer Faktoren haben eine tiefgreifende Wirkung
	darauf, ob das evolutionäre Wachstum des DORs – wie hier beschrieben – eintreten
	wird."


\subsection{Planung der Räumlichkeiten}
%%WIKIPEDIA: https://en.wikipedia.org/wiki/Hybrid_operating_room
	"Room size and preparation
	Installing a hybrid OR is a challenge to standard hospital room sizes, as not only the imaging system requires some additional space, but there are also more people in the room as in a normal OR. A team of 8 to 20 people including anasthesiologists, surgeons, nurses, technicians, perfusionists, support staff from device companies etc. can work in such an OR. Depending on the imaging system chosen, a room size of 70 square meters including a control room but excluding a technical room and the preparation areas is recommended. Additional preparations of the room necessary are 2-3mm lead shielding and potentially enforcement of the floor or ceiling to hold the additional weight of the imaging system (approximately 650–1800 kg)."
	
	"Workflow
	Planning a hybrid OR requires to involve a considerable number of stakeholders. To ensure a smooth workflow in the room, all parties working there need to state their requirements, which will impact the room design and determining various resources like space, medical, and imaging equipment.[27][28] This may require professional project management and several iterations in the planning process with the vendor of the imaging system, as technical interdependencies are complex. The result is always an individual solution tailored to the needs and preferences of the interdisciplinary team and the hospital.[22]"
	
	"Fixed C-Arms do not have these limitations, but require more space in the room. These systems can be mounted either on the floor, the ceiling, or both if a biplane system is chosen. The latter is the system of choice if pediatric cardiologists, electrophysiologists or neurointerventionalists are major users of the room. It is not recommended to implement a biplane system if not clearly required by these clinical disciplines, as ceiling-mounted components may raise hygienic issues:[29] In fact, some hospitals do not allow operating parts directly above the surgical field, because dust may fall in the wound and cause infection. Since any ceiling-mounted system includes moving parts above the surgical field and impairs the laminar airflow, such systems are not the right option for hospitals enforcing highest hygienic standards.[22] (see also[30] and,[31] both German only)
	
	"There are more factors to consider when deciding between ceiling- and floor-mounted systems. Ceiling-mounted systems require substantial ceiling space and, therefore, reduce the options to install surgical lights or booms. Nonetheless, many hospitals choose ceiling-mounted systems because they cover the whole body with more flexibility and – most importantly – without moving the table. The latter is sometimes a difficult and dangerous undertaking during surgery with the many lines and catheters that must also be moved. Moving from a parking to a working position during surgery, however, is easier with a floor-mounted system, because the C-arm just turns in from the side and does not interfere with the anesthesiologist. The ceiling-mounted system, by contrast, during surgery can hardly move to a parking position at the head end without colliding with anesthesia equipment. In an overcrowded environment like the OR, biplane systems add to the complexity and interfere with anesthesia, except for neurosurgery, where anesthesia is not at the head end. Monoplane systems are therefore clearly recommended for rooms mainly used for cardiac surgery.[22][27][29]"
	
1. http://www.ctsnet.org/article/cardiovascular-hybrid-or-clinical-technical-considerations 30.04.18	
	"Before a hybrid operating room is planned, all stakeholders in the hospital should be identified and a detailed plan of room usage developed. These high tech rooms are too costly for part time use. However, once such a facility is constructed the demand for it is increasing, because of growing indications and increased referrals (20)."
	
2. http://www.innovations-report.de//html/berichte/medizin-gesundheit/saarlaendische-shg-kliniken-setzen-hybrid-op-201121.html  30.04.18 15:45
	"Denn in Völklingen ist jetzt erstmals der Nachweis gelungen, dass die Raumluft-Anforderungen der Hygiene-Klasse Ia auch während der minimal-invasiven Eingriffe eingehalten werden können: Während in anderen Hybrid-OPs deckenmontierte Angiographie-Systeme im Betrieb die Wirksamkeit der Zuluftdecke behindern, arbeitet im Herz-Zentrum Saar der an einem bodenmontierten Roboterarm befestigte Artis zeego von Siemens reibungslos mit der Raumluft-Technik.
	„In Parkposition entsprechen zwar auch Hybrid-OPs mit deckenmontierten Angiographie-Systemen den Anforderungen der in der DIN 1946-4 festgelegten Luftreinheits-Klassen für Operationsräume. Doch sobald der medizinische Eingriff eine begleitende Bildgebung erfordert und die Angiographie-Geräte auf ihren Deckenschienen zum OP-Tisch gefahren werden, ist dort eine Störung der Raumluft-Technik unvermeidlich“, erklärt der Chefarzt des Herzzentrums, Dr. Helmut Isringhaus. Die Zuluft-Decken seien unter anderem darauf ausgelegt, möglichen Keimeintrag mit einem gleichmäßigen Luftstrom von den Patienten fernzuhalten."	

3. http://www.ctsnet.org/article/cardiovascular-hybrid-or-clinical-technical-considerations	
	"	Room size and preparation
	Interventional rooms have excellent imaging capabilities but frequently lack the prerequisites, size, and equipment required for formal operating rooms. Operating rooms meet those required standards, but usually lack high-level imaging capabilities. A hybrid suite should be larger than a standard OR and the basic principle for planning is the larger the better, because not only the imaging equipment needs sufficient space. Staff calculations have shown that in hybrid procedures 8 to 20 people are needed in the team including anesthesiologists, surgeons, nurses, technicians, perfusionists, experts form device companies and so forth (21). Expert opinions recommend for newly built ORs at least 70 m2 (4). Additional space for a control room and a technical room is mandatory adding up with washing and prep rooms to a total of approximately 150 m2 for the whole area. If a fixed C-arm system is considered, an OR size of 45 m2 is the absolute lower limit. Rebuilding in terms of lead shielding (2-3 mm) will be needed. Depending on the system it may be necessary to enforce the ceiling or the floor to hold the weight of the stand (approx. 650 – 1800 kg)
		Planning
	Planning of the hybrid room is truly an interdisciplinary task. Clinicians and technicians of all involved disciplines should define their requirements and form a responsible planning team. The concrete planning is refined in several steps by specialized architects, vendors of OR equipment, and imaging systems in a close feedback loop with the planning team. Virtual visualization of the room, visits of established hybrid rooms, and information exchange with experienced users help tremendously during the planning process. In the recent literature a couple of case studies are published for planning guidance (9, 12, 15, 18).
		Lights, Monitors, and other devices
	In general, all members of the team need access to all important information. Therefore, multiple moveable and flexible booms need to be installed in the hybrid operating room. If there are two booms to be installed, a boom on every side of the table serves the team. Collision of the ceiling-mounted displays with operating lights or other ceiling-mounted equipment should be avoided. Large displays are now available capable of showing multiple video inputs in various sizes and decreasing the need for multiple screens. A dedicated ceiling plan with all ceiling- mounted components including air condition should be drawn to ensure the function and usability of all devices."
	
	"Mono- and biplane systems
	In an overcrowded environment like an hybrid suite, biplane systems add to the complexity and interfere with anaesthesia except for neurosurgery, where anaesthesia is not at the head side. Monoplane systems (Figure 5) are therefore clearly recommended for rooms mainly used for vascular, cardiac, and orthopaedic procedures. There are certainly exceptions. If paediatric cardiologists, electrophysiologists, or neuroradiologists are important stakeholders in room usage, a biplane system may also be considered."
	
	"Table considerations
	The operating table should meet the expectations of both surgeons and interventionalists. This is in fact a special challenge, because the expectations may be mutually exclusive. Surgeons expect a table with a breakable tabletop. For imaging reasons the table has to be radiolucent and should allow coverage of the patient in a wide range. Therefore carbon fibre tabletops are used that are not breakable. Cardiovascular surgeons in general do not have very sophisticated positioning needs and are used to have fully motorized movements of the table and the tabletop. For the positioning of the patients inflatable cushions are sometimes used for positioning if no breakable table is available. Interventionalists require a floating tabletop to allow fast and precise movements during angiography and in some countries floating tabletops are among the technical requirements for performing coronary angiographies or are at least highly recommended by expert consensus (11). Floating tabletops are not available with conventional OR tables. The radiolucent area of the OR table only meet the needs in paediatric cases – a complete coverage of an adult can not be achieved with today’s systems.
	As a compromise, tables with a floating tabletop and vertical and lateral tilt are recommended (21). Special rails for mounting special surgical equipment like retractors or camera holders should be available on the table. Placing the operating table in a diagonal position in the hybrid suite may gain space. A crucial element when selecting the imaging system and table is the possibility to have access to the patient from all sides and tilting of the table both head up and down and sidewise. In order to perform 3D imaging with the operating table the C-arm has to be fully integrated with the table, because a fast and precise rotation around the patient lying in the isocentre is necessary. Breakable OR tables are currently not fully integrated and therefore 3D CT-like imaging on these tables is impossible."	
	
	
	HYGIENE HYGIENE HYGIENE HYGIENE
	1. http://www.ctsnet.org/article/cardiovascular-hybrid-or-clinical-technical-considerations 30.04.18 16:30
	"Hygiene
	Hygienic requirements differ from country to country and even among surgical disciplines with the highest standards in orthopaedic surgery. In order to guarantee highest flexibility in room usage, hospitals tend to equip all operating rooms according to the highest standards and that includes a laminar air flow ceiling. Some hospitals even require skirts around the laminar air flow field and this set up may preclude ceiling-mounted systems. In any case, ceiling mounted systems with running parts above the operating field, which are difficult to clean and interfere with the air flow by causing turbulences, are least recommended from a hygienic stand point (8)."
	
	"Expert consensus recommends floor-mounted systems for hygienic reasons. In fact, some hospitals do not allow running parts immediately above the operative field, because dust may fall down and cause infections. Despite these facts, a high number of hospitals decide to have ceiling mounted systems as these certainly cover the whole body with more flexibility and - most important - without moving the table, which is a sometimes difficult and dangerous undertaking during surgery, because many lines and catheters have to be moved alongside.  Some ceiling-mounted systems are capable of 3D imaging from a surgical position, perpendicular to the patient from both the right and left table side. Moving from a parking to a working position during surgery, however, is easier with a flexible floor-mounted system, because the C - arms just turns in from the side without interference with the anaesthesiologist, whereas ceiling-mounted systems can hardly move during hybrid procedures in the park position at the head side without colliding with the anaesthesia equipment."
	
	BLAAA BLII





