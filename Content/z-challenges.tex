\chapter{\iflanguage{ngerman}{Herausforderungen}{Challenges}}
\label{sec:overview}

\subsection{Nachteile}

Der Hybride OP-Saal weist sehr viele positive Aspekte auf, doch sind die negativen nicht zu vernachlässigen und werden deshalb im Folgenden besprochen.

Der Hybride OP-Saal wird in erster Linie durch die bildgebenden Verfahren definiert, doch wie bereits aus der Radiologie bekannt ist, müssen immer auch mögliche "Mess-, Rekonstruktions- und Modellierungsfehler berücksichtigt werden" \cite{DerDigitaleOperationssaal}. 
So können Partialvolumenartefakte Grund dafür sein, dass Tumore in der falschen Größe dargestellt werden oder bei CT Bildern die besonders kleinen Läsionen (< 1cm) überhaupt nicht in der Bildausgabe erkenntlich sind. Diese möglichen Fehler müssen in der Operationsplanung und späteren Ausführung berücksichtigt werden \cite{DerDigitaleOperationssaal}.

Hinzu kommt, dass Operationen im Hybriden OP-Saal teilweise unter laufender Röntgenkontrolle statt finden. Dieser Belastung ist dann nicht nur der Patient ausgesetzt sondern das gesamte behandelnde Team bekommt Röntgenstrahlung ab. Je nach Abstand, Winkel und Höhe zum Patienten während der Bildkontrolle, liegt die maximale Strahlungsreduktion zwischen 61 und 91\% (bei einem Patienten mit 65kg Körpergewicht). Je nach Größe und Gewicht des Patienten kann es aber auch zu einem höheren Streustrahlenanteil kommen und damit zu einer höheren Belastung für das behandelnde Team.
Nimmt man an, dass pro Operation durchschnittlich zwei 3D Scans gemacht werden, dann summiert sich die Strahlendosis für ein Teammitglied, bei 100 Operationen im Jahr, auf ungefähr 400µSv. Dies entspricht bereits 7\% der jährlichen maximalen Dosis und muss deshalb so weit wie möglich verhindert werden. Wenn immer es möglich ist, sollte das Personal bei Bildaufnahmen hinter einer Strahlenschutzwand stehen oder den OP-Saal verlassen\cite{RadiationExposure}.
%TODO, noch in \cite{HybridOR} schauen --> radiation dose and dose reduction

Ein anderes Problem entsteht beispielsweise bei der Tumorentfernung im Gehirn mit intraoperativem MRT. Um kleine Veränderungen (wie Brain Shift) frühzeitig erkennen und darauf reagieren zu können, sind viele regelmäßige Bildaufnahmen nötigt. Gleichzeitig muss für jede Bildaufnahme die Operation unterbrochen und für jedes Bild ein gewisser Zeitaufwand aufgebracht werden. Für ein iMR-Bild muss etwa mit einer Aufnahmezeit von circa 15 Minuten gerechnet werden \cite{BrainShiftInTumorResection}. Jede Aufnahme führt damit auch zu einer Verlängerung der Operationszeit und es muss somit ein Kompromiss zwischen Zeitaufwand und Bildhäufigkeit gefunden werden.\\
Alternativ besteht in diesem Fall immer noch die Möglichkeit, eine schlechtere Bildqualität in Kauf zu nehmen und dafür nur eine Aufnahmezeit von 5 Minuten, wenn man stattdessen zum US-Gerät greift. Unter diesen Umständen sollte man aber nicht vergessen, dass Ultraschall, im Gegensatz zu Magnetresonanztomografie, nicht kontaktlos verwendet werden kann, was somit immer auch ein höheres Infektionsrisiko birgt \cite{BrainShiftInTumorResection}.

Ein Punkt der wohl vielen Krankenhäusern Kopfzerbrechen bereitet, werden wohl die hohen Investitions- und Wartungskosten eines Hybriden OP-Saals sein. Die benötigte Raumgröße und Ausrüstung führen dazu, dass ein Hybrider OP-Saal in der Anschaffung mehr als doppelt so teuer und in der Wartung fast doppelt so teuer ist wie ein konventioneller OP-Saal \cite{HybridOR}. 
Ob diese Kosten gerechtfertigt und lohnend sind wird im nächsten Kapitel in der Kosten-Nutzen Analyse erörtert.
%TODO nächstes Kaptiel?? oder übernächstes eher oder Herausforderungen???

\subsection{Kosten- Nutzen Verhältnis}

Eine der wichtigen Fragen die noch geklärt werden muss, ist ob ein Hybrider OP-Saal tatsächlich den Mehraufwand an Kosten und Umstrukturierung lohnt. 
Ist es wirklich gerechtfertigt circa das doppelte an Investitions- und Wartungskosten in einen OP-Saal zu investieren \cite{ORofTheFuture}? Und wie viel Geld ist ein Menschenleben denn wert? Was darf ein OP-Saal mehr kosten wenn die Sterblichkeitsrate einer Operation um beispielsweise 1\% gesenkt wird? Nehmen wir an, bei diesem Vorgang kommen 50\% aller Patienten ums Leben, bei einer Verminderung von 1\% stirbt bei 100 Menschen eine Person weniger und bei 1000 Menschen sind es schon 10 die überleben. Darf der OP-Saal dafür bereits 100\% mehr kosten?
Gleichzeitig hat sich herausgestellt, dass neue medizinische Technologien meist direkt positiv aufgenommen werden, obwohl ein Mehrwert dieser noch gar nicht bewiesen wurde. Und in den letzten Jahren sind die Gesundheitskosten erheblich gestiegen, was sehr eng mit Digitalisierung des OPs zusammenhängt \cite{DerDigitaleOperationssaal}.

Weil die oben genannten Fragestellungen nicht wirklich beantwortet werden können, soll erstmal genau herausgearbeitet werden, welchen Mehrwert der Hybride OP-Saal tatsächlich bringt. Weil die Vorteile des Hybriden OP-Saals bereits im ersten Kapitel herausgearbeitet wurden, wird hier deshalb nur auf die Sterblichkeitsrate, Operationszeit und Kosten eingegangen.

Im Anwendungsfall des Bauchaortenaneurysma konnte eine Operationszeiteinsparung von 23,5 Minuten (von 120 auf 96,5 Minuten), mit einem Hybriden OP-Saal gegenüber einem konventionellen mit C-Bogen, erreicht werden. Die Zeiteinsparung führt zusätzlich auch zu einer Kosteneinsparung der Prozesskosten und beträgt 276,17€ weniger pro durchgeführte Operation.
Dieses Ergebnis muss jedoch kritisch betrachtet werden, da die Studie auf Ergebnissen eines konventionellen OPs mit 97 Patienten von 2007 bis 2010 und beim Hybriden mit 50 Patienten von 2012 bis 2015 durchgeführt wurde. Ein positiver Trend ist dennoch sehr wohl zu verzeichnen, da bei den Patienten sehr darauf geachtet wurde ähnliche Charakteristika zu besitzen und auch das Operationsteam wurde so ausgewählt \cite{HybriderVsKonventioneller}.
Es muss jedoch immer bedacht werden, dass zwei so komplexe Systeme wie hier betrachtet, kaum miteinander verglichen werden können. Es wird trotz enormer Ähnlichkeiten immer einen gewissen Unterschied zwischen den verglichenen Patienten und Teams geben \cite{DerDigitaleOperationssaal}.

%TODO Herausforderungen: auf  Brain Shift, Aortenaneurysma, Tumorentfernung und Herzchirurgie Todesrate wieder drauf eingehen wenn möglich

%\cite{HybridOR} costs and return on investment

%Kaptiel 1 DOR

%The tumors were microscopically completely removed in 14 out of 16 cases. Thus, iMR image
%data compensate for the effects of brain shift with a high degree of accuracy. Updating the neuronavigation system
%with intraoperative MR images seems to be the most reliable way to compensate for intraoperative brain shift
%\cite{BrainShiftInTumorResection}


%It is anticipated that with the creation of a super subspecialty,
%cardiovascular hybrid surgery, many of these hybrid
%procedures could be done without the need for collaboration
%of specialists from various medical fields, as is
%the current practice. \cite{ORofTheFuture}

\subsection{Planung der Räumlichkeiten}

Die Raumplanung eines Hybriden OP-Saals ist ein sehr komplexer Prozess, da er mehreren Fachbereichen gerecht werden muss die unterschiedliche und teilweise auch sich gegenseitig ausschließende Ansprüche haben. Aus diesem Grund müssen alle fachbereichsübergreifenden Chirurgen aber auch Anästhesisten, Arzthelfer und Techniker in den Planungsprozess miteinbezogen werden.
Gleichzeitig ist durch die erhöhte Anzahl an Mitgliedern (8 bis 20 Personen) im Operationsteam eine Raumgröße von circa 70m² empfehlenswert. Mit Kontroll-, Technik und Vorbereitungsraum müssen dann insgesamt mit circa 150m² gerechnet werden. Je nachdem ob die Geräte und Systeme wie der C-Bogen über eine Decken- oder Bodenbefestigung angebracht werden, müssen diese einem Gewicht von 650 bis 1800kg standhalten. Gleichzeitig dürfen sich die bildgebenden Systeme, Monitore, Beleuchtungsanlagen und das Personal nicht in die Quere kommen. Durch die große Raumgröße und Montage von Geräten an der Decke, wird zusätzlich noch die Einhaltung der Hygienevorschriften erschwert \cite{TechnicalConsiderations}.\\
Aus den genannten Gründen muss normalerweise mit einer relativ langen Planungsphase für einen Hybriden OP-Saal gerechnet werden, um allen Ansprüchen einigermaßen gerecht zu werden und einen zukunftsfähigen OP-Saal zu errichten.

%TODO vielleicht noch Workflow einfügen; und flexibles Raumlayout, also dass man vieles bewegen also wegschieben kann \cite{HybridOR}


 





