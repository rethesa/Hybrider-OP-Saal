\chapter{\iflanguage{ngerman}{Herausforderungen}{Challenges}}
\label{sec:overview}

\subsection{Kosten- Nutzen Verhältnis}

Eine der wichtigen Fragen die noch geklärt werden muss, ist ob ein Hybrider OP-Saal tatsächlich den Mehraufwand an Kosten und Umstrukturierung lohnt. 
Ist es wirklich gerechtfertigt circa das doppelte an Investitions- und Wartungskosten in einen OP-Saal zu investieren \cite{ORofTheFuture}? Und wie viel Geld ist ein Menschenleben denn wert? Was darf ein OP-Saal mehr kosten wenn die Sterblichkeitsrate einer Operation um beispielsweise 1\% gesenkt wird? Nehmen wir an, bei diesem Vorgang kommen 50\% aller Patienten ums Leben, bei einer Verminderung von 1\% stirbt bei 100 Menschen eine Person weniger und bei 1000 Menschen sind es schon 10 die überleben. Darf der OP-Saal dafür bereits 100\% mehr kosten?
Gleichzeitig hat sich herausgestellt, dass neue medizinische Technologien meist direkt positiv aufgenommen werden, obwohl ein Mehrwert dieser noch gar nicht bewiesen wurde. Und in den letzten Jahren sind die Gesundheitskosten erheblich gestiegen, was sehr eng mit Digitalisierung des OPs zusammenhängt \cite{DerDigitaleOperationssaal}.

Weil die oben genannten Fragestellungen nicht wirklich beantwortet werden können, soll erstmal genau herausgearbeitet werden, welchen Mehrwert der Hybride OP-Saal tatsächlich bringt. Weil die Vorteile des Hybriden OP-Saals bereits im ersten Kapitel herausgearbeitet wurden, wird hier deshalb nur auf die Sterblichkeitsrate, Operationszeit und Kosten eingegangen.



Im Anwendungsfall des Bauchaortenaneurysma konnte eine Operationszeiteinsparung von 23,5 Minuten (von 120 auf 96,5 Minuten), mit einem Hybriden OP-Saal gegenüber einem konventionellen mit C-Bogen, erreicht werden. Die Zeiteinsparung führt zusätzlich auch zu einer Kosteneinsparung der Prozesskosten und beträgt 276,17€ weniger pro durchgeführte Operation.
Dieses Ergebnis muss jedoch kritisch betrachtet werden, da die Studie auf Ergebnissen eines konventionellen OPs mit 97 Patienten von 2007 bis 2010 und beim Hybriden mit 50 Patienten von 2012 bis 2015 durchgeführt wurde. Ein positiver Trend ist dennoch sehr wohl zu verzeichnen, da bei den Patienten sehr darauf geachtet wurde ähnliche Charakteristika zu besitzen und auch das Operationsteam wurde so ausgewählt \cite{HybriderVsKonventioneller}.
Es muss jedoch immer bedacht werden, dass zwei so komplexe Systeme wie hier betrachtet, kaum miteinander verglichen werden können. Es wird trotz enormer Ähnlichkeiten immer einen gewissen Unterschied zwischen den verglichenen Patienten und Teams geben \cite{DerDigitaleOperationssaal}.

%TODO Herausforderungen: auf  Brain Shift, Aortenaneurysma, Tumorentfernung und Herzchirurgie Todesrate wieder drauf eingehen wenn möglich

%Kaptiel 1 DOR

%The tumors were microscopically completely removed in 14 out of 16 cases. Thus, iMR image
%data compensate for the effects of brain shift with a high degree of accuracy. Updating the neuronavigation system
%with intraoperative MR images seems to be the most reliable way to compensate for intraoperative brain shift
%\cite{BrainShiftInTumorResection}


%It is anticipated that with the creation of a super subspecialty,
%cardiovascular hybrid surgery, many of these hybrid
%procedures could be done without the need for collaboration
%of specialists from various medical fields, as is
%the current practice. \cite{ORofTheFuture}

\subsection{Planung der Räumlichkeiten}





