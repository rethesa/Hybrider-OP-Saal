\chapter{\iflanguage{ngerman}{Herausforderungen}{Challenges}}
\label{sec:overview}

Der Hybride OP-Saal bringt Vorteile für die zukünftige Entwicklung des OP-Saals. Bei der Planung, Umsetzung und Vernetzung kommen Herausforderungen auf, die bewältigt werden müssen. Nicht alle Aspekte sind positiv und es soll validiert werden, ob sich ein Hybrider OP-Saal tatsächlich lohnt.

\subsection{Nachteile}

\begin{figure} [H]
	\includegraphics[scale = 0.7]{Content/Pictures/partial.png}
	\caption{Beispiel für Partialvolumenartefakte: (links) Abbild eines Tumors, (mitte) das finale Bild und (rechts) alle Voxel in denen Grauwerte des Tumors verzeichnet wurden \cite{DerDigitaleOperationssaal}.}
	\label{fig:partial}
\end{figure}

Der Hybride OP-Saal wird in erster Linie durch die bildgebenden Verfahren definiert. Wie aus der Radiologie bekannt müssen auch mögliche \glqq Mess-, Rekonstruktions- und Modellierungsfehler berücksichtigt werden\grqq{} \cite{DerDigitaleOperationssaal}. So können Partialvolumenartefakte (siehe Abb. \ref{fig:partial}) Grund dafür sein, dass Tumore in der falschen Größe dargestellt oder kleine Läsionen (< 1cm) nicht in den CT-Bildern abgebildet werden. Diese Fehler müssen in der komplexen und zeitaufwendigen Operationsplanung und späteren Durchführung berücksichtigt werden \cite{DerDigitaleOperationssaal}.

Hinzu kommt, dass Operationen im Hybriden OP-Saal teilweise unter laufender Röntgenkontrolle stattfinden. Diese Strahlenbelastung betrifft nicht nur den Patienten, sondern das gesamte behandelnde Team. Dieses wird der durch den Patienten verursachten Streustrahlung ausgesetzt. Je nach Abstand, Winkel und Höhe zum Patienten während der Bildkontrolle wird eine anwesende Person 9 bis 39\% der Strahlung ausgesetzt, die der Patient ausgesetzt wird \cite{RadiationExposure}. Abhängig von Größe und Gewicht des Patienten kann es aber auch zu einem höheren Streustrahlenanteil kommen und damit zu einer höheren Belastung für das behandelnde Team. Ist eine Person bei 100 Operationen mit je zwei 3D Scans pro Jahr anwesend, wird diese bereits 7\% der maximalen Jahresdosis ausgesetzt \cite{RadiationExposure}. Aus diesem Grund muss das Personal bei der Bildaufnahme hinter einer Strahlenschutzwand stehen oder wenn möglich den OP-Saal verlassen.

Ein anderes Problem entsteht bei der Tumorentfernung im Gehirn mit Hilfe von iMR. Um Veränderungen (wie Brain Shifts) frühzeitig erkennen und entsprechend reagieren zu können, sind regelmäßige Bildaufnahmen nötig. Für jede Bildaufnahme muss jedoch die Operation unterbrochen und ein Zeitaufwand für die Aufnahme aufgebracht werden \cite{BrainShiftInTumorResection}. Jede Aufnahme führt damit zu einer Verlängerung der Operationszeit und es muss ein Kompromiss zwischen Zeitaufwand und Bildhäufigkeit gefunden werden.\\
Alternativ kann stattdessen das US-Gerät verwendet werden. US ist aber im Gegensatz zu Magnetresonanztomografie nicht kontaktlos einsetzbar, was ein höheres Infektionsrisiko mit sich bringt \cite{BrainShiftInTumorResection}.

\subsection{Planung der Räumlichkeiten}

Die Raumplanung eines Hybriden OP-Saals ist ein komplexer Prozess (Abb. \ref{fig:roomplanning}), da er mehreren Fachbereichen gerecht werden muss. Die Bereiche haben unterschiedliche und teilweise sich gegenseitig ausschließende Ansprüche \cite{TechnicalConsiderations}. Aus diesem Grund müssen alle fachbereichsübergreifenden Chirurgen aber auch Anästhesisten, Arzthelfer und Techniker in den Planungsprozess miteinbezogen werden.
Durch die erhöhte Anzahl an Mitgliedern (8 bis 20 Personen) im Operationsteam ist eine Raumgröße von circa 70m² empfehlenswert. Mit Kontroll-, Technik- und Vorbereitungsraum müssen insgesamt mit circa 150m² gerechnet werden. Abhängig davon, ob die Geräte und Systeme wie der C-Bogen über eine Decken- oder Bodenbefestigung angebracht werden, müssen diese einem Gewicht von 650 bis 1800kg standhalten. Gleichzeitig dürfen die bildgebenden Systeme, Monitore, Beleuchtungsanlagen und das Personal nicht kollidieren. Durch die Raumgröße und Montage von Geräten an der Decke wird zusätzlich die Einhaltung der Hygienevorschriften erschwert \cite{TechnicalConsiderations}.\\ 
Aus den genannten Gründen muss normalerweise mit einer relativ langen Planungsphase für einen Hybriden OP-Saal gerechnet werden, um allen Ansprüchen einigermaßen gerecht zu werden und einen zukunftsfähigen OP-Saal zu errichten.

\begin{figure} [t!]
	\includegraphics[scale = 0.7]{Content/Pictures/roomplanning.png}
	\caption{Beispielhaftes Layout für die Planung eines Hybriden OP-Saals \cite{HybridOR}.}
	\label{fig:roomplanning}
\end{figure}

\subsection{Kosten- Nutzen Verhältnis}

Eine der Fragen, die noch geklärt werden müssen, ist, ob ein Hybrider OP-Saal tatsächlich den Mehraufwand an Kosten und Umstrukturierung lohnt. Die benötigte Raumgröße und Ausrüstung führen dazu, dass ein Hybrider OP-Saal in der Anschaffung mehr als doppelt so teuer und in der Wartung fast doppelt so teuer wie ein Konventioneller OP-Saal ist \cite{HybridOR}. Gleichzeitig hat sich herausgestellt, dass neue medizinische Technologien positiv aufgenommen werden, obwohl ein Mehrwert dieser nicht bewiesen wurde. Dies und die Digitalisierung des Operationssaals haben in den letzten Jahren zu erheblichen Steigerungen der Gesundheitskosten beigetragen \cite{DerDigitaleOperationssaal}. \\
Welchen Mehrwert ein Hybrider OP-Saal tatsächlich bringt wird in Bezug auf Kosten und Operationszeit überprüft.

Im Anwendungsfall des Bauchaortenaneurysma konnte eine Operationszeiteinsparung von 23,5 Minuten (von 120 auf 96,5 Minuten) mit einem Hybriden OP-Saal gegenüber einem Konventionellen mit C-Bogen erreicht werden. Die Zeiteinsparung führt zu einer Einsparung der Prozesskosten um 276,17€ pro durchgeführter Operation. Bei dieser Studie wurde die Ergebnisse eines Hybriden OP-Saals von 2012 bis 2015 mit einem Konventionellen OP-Saal von 2007 bis 2012 verglichen. Im ersten Fall wurden Daten von 50 Patienten und im zweiten von 97 verwendet. Ein positiver Trend ist dennoch zu verzeichnen, da bei den Patienten und dem Operationsteam auf ähnliche Charakteristika geachtet wurde \cite{HybriderVsKonventioneller}.\\
Durch die Operationszeiteinsparung im Hybriden OP-Saal ist also tatsächlich eine Refinanzierung möglich \cite{HybriderVsKonventioneller}. Hinzu kommen positive Ergebnisse ohne direkten Vergleichswert, wie bei der Tumorentfernung. Bei 14 aus 16 Fällen konnten die Tumore komplett entfernt und auftretende Brain Shifts frühzeitig erkannt werden \cite{BrainShiftInTumorResection}.

Da Studien zum Vergleich von einem Hybriden und Konventionellen Operationssaal zum einen kaum vorliegen und zum anderen schwierig durchzuführen und zu validieren sind, lässt sich die Frage des tatsächlichen Nutzens sehr schwer beantworten. Auch der Vergleich von Todesraten ist kein aussagekräftiges Ergebnis, da Überleben auch mit Lebensqualität zusammenhängt\cite{HybriderVsKonventioneller}. \\
Weitere Faktoren, die einen Einfluss auf den Nutzen haben, sind die gefühlte und tatsächliche Sicherheit von Patienten und Chirurgen. Die psychologische Seite ist jedoch noch schwieriger zu bewerten als das tatsächliche objektive Ergebnis \cite{DerDigitaleOperationssaal}.

Trotzdem lässt sich festhalten, dass der Hybride OP-Saal seine Vorteile mit sich bringt. Nicht in allen medizinischen Bereichen wird er einen gleichgroßen Nutzen zur Geltung bringen, doch es eröffnen sich neue Möglichkeiten zur Operationsdurchführung \cite{ORofTheFuture}.





 





