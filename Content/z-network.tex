\chapter{\iflanguage{ngerman}{Netzwerkkommunikation}{Network communication}}
\label{sec:overview}

\subsection{ESIS - Surgical Deck}
1. Der digitale Operationssaal- Kapitel 2
	"Die Grundlage des Surgical Deck ist eine vollständig digitale Datenbasis aller relevanten
	Systeme, die hier in der Gesamtheit als ESIS bezeichnet wird. ESIS kann dabei mit
	einem IP-Netzwerk (Internet Protocol Network) verglichen werden, dass die anfallenden
	Daten aufnimmt, bündelt, weiterleitet und übergibt. ESIS unterscheidet grundsätzlich
	zwischen unbehandelten („rohen“) Daten, wie dem HD-live-Signal der chirurgischen
	Kamera und kalkulierten („bearbeiteten“) Daten, die in irgendeiner Form
	von einem Rechnersystem verarbeitet werden, bevor sie dem Operateur angeboten
	werden [7]. Sicherheitskritische und/oder echtzeitfähige Daten müssen von einem
	ESIS besonders definiert und behandelt werden. ESIS basiert im hier dargestellten
	Surgical Deck auf dem Storz Communication Bus (SCB), ab der Version 2.0 in Erweiterung
	mit einem IP-Bus (OR1 Fusion der KARL STORZ GmbH und Co. KG) [4]. Einzelne
	(insbesondere echtzeitkritische) Funktionen werden durch weitere Bussysteme, wie
	den Open Navigation Bus (ONB) ergänzt [8]. Grundsätzlich ist ESIS bereits für weitere
	Datenformate vorbereitet und kann perspektivisch auch auf anderen Bussystemen
	aufbauen (z. B. IHE) [9]. Viele Teilsysteme innerhalb (z. B. Lichtquelle, chirurgischer
	Motor, OP-Tisch, Navigationssystem) und außerhalb (z. B. Bildarchiv/PACS, Elektronische
	Patientenakte) des Surgical Deck verfügen bereits heute über eine entsprechende
	Schnittstelle (SCB, ONB) oder schnittstellenkompatible Datenformate (DICOM, HL7)."