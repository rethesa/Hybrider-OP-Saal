\chapter{\iflanguage{ngerman}{Netzwerkkommunikation}{Network communication}}
\label{sec:overview}

\subsection{Kommunikations- und Netzwerkarchitekturen}
1. Der digitale Operationssaal- Kapitel 3
	". Hierbei handelt es sich jedoch meist um geschlossene,
	proprietäre Systeme, so dass eine Integration von Komponenten von Drittanbietern
	nicht oder nur mit hohem Aufwand (Zeit, Kosten) möglich ist. Um diese Einschränkungen
	zu überwinden, werden herstellerübergreifende Standards und Konzepte für
	eine modulare Integration von Medizinprodukten im OP benötigt. Die heute verfügbaren
	Standards und Konzepte decken die Anforderungen an eine modulare Integration
	nur unzureichend ab. Um diese Hindernisse zu überwinden, wird daher ein
	neuartiges Integrationskonzept für den OP basierend auf einer service-orientierten
	Architektur vorgestellt.
	Die verschiedenen Medizinprodukte stehen dabei in einem Netzwerk als Services
	zur Verfügung bzw. greifen auf Services zu. Eine Verwaltung des Gesamtsystems
	erfolgt über zentrale Komponenten, die über standardisierte Schnittstellen und den
	Open-Surgical-Communication-Bus als zentrales Transportmedium mit den Einzelkomponenten
	kommunizieren. Standardisierte Schnittstellen, die ausgehend von der
	IEEE-11073-Nomenklatur entworfen wurden, ermöglichen den modularen und flexiblen
	Austausch von Komponenten"
	
	"In modernen Operationssälen (OP) nimmt die Komplexität der Mensch-Technik-
	Interaktion durch den Einsatz von technischen Geräten und computergestützten
	Assistenzsystemen mehr und mehr zu. Um die Handhabung zu erleichtern und den
	Workflow zu optimieren, bieten verschiedene Medizintechnikhersteller Integrationslösungen
	für den OP an, sogenannte Integrated Operating Room Systems (IORS)
	[1, 2]. Das primäre Ziel ist, die Anzahl unterschiedlich gestalteter Mensch-Maschine-
	Schnittstellen im OP zu reduzieren und einen zentralen Zugriff auf die Bedienung
	aller relevanten Geräte zu ermöglichen.
	Ein weiteres wichtiges Ziel ist es, die Kommunikation und den Datenaustausch
	zwischen den Geräten zu vereinfachen bzw. zu ermöglichen [3, 4]."
	
	"Dies kann zum einen zu einer Einschränkung der Behandlungsoptionen für den
	individuellen Patienten führen, da die Integration und Anwendung von eventuell
	optimaleren Systemkomponenten konkurrierender Anbieter nicht ermöglicht wird
	[6]. Zum anderen ergeben sich durch die Abhängigkeiten der Kliniken von den Herstellern
	der IORS auch potenziell ökonomische Nachteile. Die aktuellen Probleme
	integrierter OP-Systeme sind in Abbildung 3.1 zusammengefasst.
	Eine modulare und herstellerübergreifende Vernetzung wird durch fehlende
	Schnittstellen- und Protokollstandards verhindert bzw. erschwert [7]. Mit DICOM, HL7
	und IHE existieren Standards zum Datenaustausch im medizinischen Umfeld, jedoch
	decken sie nur einen Teil der Aspekte des Informationsaustausches im Operationssaal
	ab.
%%	-->	 Modularität und Flexibilität ist eingeschränkt
%%	 Innovationen und Optimierungen werden behindert
%%	 Belastungen und Risiken durch eingeschränkte Gebrauchstauglichkeit"	
	
	"Die IEC 80001-1 (Application of risk management for IT-networks incorporating
	medical devices) soll zukünftig eine Grundlage für die modulare, herstellerübergreifende
	Integration von Medizinprodukten in klinische IT-Netzwerke bilden [10]. Sie
	legt die Rollen und Verantwortlichkeiten fest und beschreibt die Maßnahmen des
	Risikomanagements im Gesamtkontext.
	Zur technischen Umsetzung entsprechender Integrationslösungen werden offene
	Architekturen und Standards für Plug and Play-Lösungen benötigt. Im Folgenden wird
	ein kurzer Überblick über kommerzielle IORS gegeben und anschließend das Konzept
	einer offenen, modularen, herstellerunabhängigen Architektur für die Integration
	von medizinischen Geräten im OP vorgestellt, das derzeit im Rahmen des Projektes
	smartOR exemplarisch umgesetzt wird."
	
	"Das smartOR-Integrationsframework besteht aus folgenden Hauptkomponenten:
	ein Open Surgical Communication Bus (OSCB), der durch ein IP-basiertes Netzwerk
	umgesetzt und gegebenenfalls durch einen zusätzlichen zentralen Kommunikationsserver
	unterstützt wird, sowie medizinischen Geräten und Komponenten bzw. Hardund
	Softwaremodulen, die als Service Provider und/oder Service Consumer Dienste
	über den OSCB nutzen und/oder bereitstellen.
	Der OSCB stellt das zentrale Übertragungsmedium und damit die gemeinsame
	Kommunikationsgrundlage für die Geräte dar. Um eine Kommunikation zu ermöglichen,
	ist die Verwendung von standardisierten und offenen Schnittstellen für die
	verschiedenen Medizingeräte ein zentraler Aspekt. In der SOA werden die Geräte und
	ihre Funktionen gekapselt und so ihre Gerätefunktionen als Dienste über standardisierte
	Schnittstellen via OSCB anderen Geräten angeboten."
	
\subsection{Sicherheitsprobleme durch Netzwerkkommunikation}
1. Der digitale Operationssaal- Kapitel 3

	"Die einzige Verbindung zwischen dem OP-IT-Netzwerk und dem Klinik-ITNetzwerk
	wird über ein Gateway realisiert. Eine Kommunikation mit Systemen, die
	nicht dem Netzwerk angehören (etwa KIS- oder PACS-Systeme), darf ausschließlich
	über das Gateway erfolgen, das die beiden Netzwerke entsprechend den jeweiligen
	Sicherheitsanforderungen logisch trennt. Dazu gehört der Einsatz einer Firewall, um
	unerwünschten Datenverkehr sowohl in das als auch aus dem Netzwerk zu blockieren
	und Angriffe von außen (z. B. durch Malware oder Hacker) abzuwehren [22]."
	
	"Durch die Vernetzung und den Austausch von Daten über ein Local Area Network
	werden die medizinischen Geräte enger in die IT-Welt eingebunden und sind von
	Gefahren betroffen. Es müssen Verfahren zum Schutz des klinischen IT-Netzwerkes
	implementiert und validiert werden. Als grundlegende Schutzmaßnahme wurde ein
	Gateway vorgesehen, was jedoch alleine nicht ausreichend ist. In einem Netzwerk von
	Medizinprodukten existieren grundsätzlich verschiedene Interpretationen des Begriffs
	„Sicherheit“, die sich inhaltlich nur wenig überschneiden. Zum einen ist damit die
	unmittelbare Sicherheit von Patienten, OP-Personal und sonstigen Personen angesprochen,
	auf die das Gerät bei der Ausübung seiner Funktion Einfluss nimmt. Beispielsweise
	kann ein Anästhesiegerät durch falsche Einstellungen oder eine Spritzenpumpe
	durch falsche Dosierung intolerablen Schaden verursachen. Dieses Problem besteht
	jedoch bereits wenn das Gerät nicht vernetzt ist. Durch die Vernetzung ergeben sich
	neue Möglichkeiten des Zugriffs. Es muss unter anderem sichergestellt werden, dass
	ein Medizingerät nur authentisierte, verifizierte und validierte Befehle entgegennimmt
	und ausführt. Insbesondere muss sichergestellt werden, dass das befehlende Gerät zur
	Abgabe der entsprechenden Befehle berechtigt ist.
	Die geforderten Eigenschaften [37] zur Informationssicherheit"
	
	"Im Hinblick auf den Datenschutz ist eine vertrauliche
	Übermittlung von sensiblen Daten wie Patientenidentität, Vitalparameter,
	Krankengeschichte und Medikation unerlässlich. Daher sind die beteiligten Geräte
	vor unberechtigtem Zugriff und vor Ausspähen zu schützen. Ein möglicher Ansatz ist
	die Erweiterung WS-Security für SOAP aus der WS-*-Familie. WS-Security erlaubt es,
	SOAP-Nachrichten zu signieren, um die Integrität der Nachricht nachzuweisen; Nachrichten
	können verschlüsselt übertragen werden, um Vertraulichkeit sicherzustellen
	und es kann über Sicherheitstoken die Identität des Absenders bewiesen werden."
	
	BLAA BLKI