\chapter{\iflanguage{ngerman}{Entwicklung des Hybriden OP-Saals}{Development}}
\label{sec:overview}

\subsection{Vorteile des Hybriden OP-Saals}
http://www.innovations-report.de/html/berichte/medizintechnik/operieren-im-op-der-zukunft.html 30.04.18 16:00
	"Als erstes Spital der Schweiz verfügt das Inselspital über einen hochmodernen OP-Bereich, der interdisziplinär von allen chirurgischen Fachgebieten genutzt werden kann. Mit sämtlichen modernsten Bildgebungstechniken ausgerüstet, werden bei komplizierten Operationen Qualitätskontrollen während des Eingriffs möglich sein."
	
	"Genutzt wird der Hochpräzisions-OP von der Neurochirurgie, der Orthopädie, der Viszeral- und der Kieferchirurgie, der HNO-Chirurgie und der Urologie, jeweils in enger Zusammenarbeit mit der Neuroradiologie, der Radiologie und der Anästhesie. Das flexible und ökonomische Mehrraumkonzept des Hochpräzisions-OP ermöglicht den beteiligten Fachgebieten grosse Synergien in der Nutzung der kostenintensiven Bildgebung."
	
	"Mehr Sicherheit und bessere Ergebnisse für Patientinnen und Patienten
	„Mit den intraoperativen tomografischen und dreidimensionalen Bildgebungs- und den damit verbundenen Navigationstechniken im Hochpräzisions-OP können die Chirurginnen und Chirurgen direkt während der Operation – und nicht erst danach – Bildkontrollen durchführen und allenfalls Korrekturmassnahmen ergreifen“, erläutert Prof. Dr. med. Jan Gralla, Institutsdirektor und Chefarzt des Universitätsinstituts für Diagnostische und Interventionelle Neuroradiologie. Es gibt viele komplizierte Operationen, bei denen eine Bildkontrolle des Operationsfortschrittes sinnvoll ist."

http://www.innovations-report.de/html/berichte/medizintechnik/dresdner-uniklinikum-nimmt-high-end-hybrid-op-in-betrieb.html 30.04.18 16:00	
	"Zudem sorgt die komplexe Steuerung der Angiographieanlage dafür, dass sich Patienten dem Eingriff entsprechend optimal positionieren lassen. Eine weitere Besonderheit stellt der auf von einem Roboterarm geführte C-Bogen dar. Durch dessen hochbewegliche, intelligente Führung lassen sich Röntgenbilder aus jeweils optimalen Positionen erstellen."
	
	"„Mit dem neuen Hybrid-OP stoßen wir die Tür zu einer neuen Epoche der Chirurgie weit auf. Die intelligente Vernetzung moderner Bildgebungsverfahren mit allen anderen Geräten sorgt dafür, dass wir noch präziser, sicherer und dabei noch schonender operieren können. Die jetzt in Betrieb genommene High-End-Angiographieanlage ist dabei nur ein erster Schritt. Von ihr werden nicht ausschließlich Gefäßpatienten profitieren, sondern auch Krebskranke“, sagt Prof. Jürgen Weitz, Direktor der Klinik für Viszeral-, Gefäß- und Thoraxchirurgie. "
	
	"Hauptnutzer des neuen High-End Hybrid-OP in Haus 59 ist das gefäßchirurgische Team um Prof. Christian Reeps. Ein Schwerpunkt des Gefäßchirurgen bilden Eingriffe bei Patienten mit lebensgefährlichen Erkrankungen der Hauptschlagader – wie zum Beispiel einem Aortenaneurysma. Hier spielen schonende Verfahren mit Implantationen großer maßgefertigter Gefäßprothesen eine wichtige Rolle. Zwar können komplexe Aneurysmen auch in einer offenen Operation mit konventionellen Prothesen behandelt werden.
	
	Senden	
	Doch über die Leistenschlagader eingeführte Stents ermöglichen es, auch betagte Menschen erfolgreich zu versorgen, für die eine offene Operation nicht mehr in Frage kommt. Das minimalinvasiv über einen Katheter unter Röntgenkontrolle vorgenommene Setzen von Stentprothesen kann hochkomplex sein. Hier spielt der neue Hybrid-OP seine Stärken voll aus, in dem Patienten nicht nur optimal operiert werden können, sondern sich während des Eingriffs auch hochauflösende zwei- oder dreidimensionale Bilder anfertigen lassen.
	
	Diese können während der OP mit den im Vorfeld der Intervention angefertigten Bildern eines Computertomographen zusammengeführt werden. Darüber hinaus gehört zu der Angiographieanlage eine Navigationssoftware, die den Operateuren dabei hilft, die Stents auf dem Weg durch die Gefäße sicher und schnell an ihr Ziel zu bringen."

http://www.ctsnet.org/article/cardiovascular-hybrid-or-clinical-technical-considerations	
	" Combining interventions and surgery into a single therapeutic procedure leads to reduction of complexity, cardiopulmonary bypass time, risk, and improved outcomes. "
	
	"The hybrid operating suite itself represents an extremely complex working environment that demands careful planning by all stakeholders. Bundling of clinical, technical, and architectural expertise as well as a realistic view on what is achievable is key for a successful hybrid OR project. Due to wide variations in utilization generic recommendations only are of limited use for these highly individual rooms and certainly cannot replace the diligent work of the project team. However, once the room is successfully established it really transforms surgical techniques and paves the way for revolutionary new minimally-invasive therapies."

http://www.innovations-report.de/html/berichte/studien-analysen/nachfrage-hybridloesungen-stimuliert-europamarkt-214570.html 30.04.18 17:30	
	"Hybridsysteme kombinieren OP-Ausrüstungen mit großen Diagnoseeinheiten und können von vielen medizinischen Fachrichtungen genutzt werden. Die hohen Kosten für diese Hybridlösungen könnten jedoch mittelfristig ein Hindernis darstellen. "
\subsection{Wichtige Einsatzgebiete}
- Neurochirugie, Thoraxchirugie und endobronchiale Eingriffe, Biopsie
	https://www.revolvy.com/main/index.php?s=Hybrid%20operating%20room&item_type=topic 01.05.18 18:00
	"Clinical applications
	Hybrid operating rooms are currently used mainly in cardiac, vascular and neurosurgery, but could be suitable for a number of other surgical disciplines."

WIKIPEDIA: https://en.wikipedia.org/wiki/Hybrid_operating_room
	"A hybrid operating room is a surgical theatre that is equipped with advanced medical imaging devices such as fixed C-Arms, CT scanners or MRI scanners.[1] These imaging devices enable minimally-invasive surgery. Minimally-invasive surgery is intended to be less traumatic for the patient and minimize incisions on the patient and perform surgery procedure through one or several small cuts.
	Though imaging has been a standard part of the OR for a long time in the form of mobile C-Arms, ultrasound and endoscopy, these new minimally-invasive procedures require imaging techniques that can visualize smaller body parts such as thin vessels in the heart muscle and can be facilitated through intraoperative 3D imaging.[1]"
	
	"Functional imaging in the OR
	Improvements of the C-Arm technology nowadays also enable perfusion imaging and can visualize parenchymal blood volume in the OR. To do that, rotational angiography (3D-DSA) is combined with a modified injection protocol and a special reconstruction algorithm. The blood flow can then be visualized in the course of time. This can be useful in the treatments of patients suffering from ischemic stroke.[21]
	
	Imaging techniques with a CT
	A CT system mounted on rails can be moved into and out of an OR to support complex surgical procedures, such as brain, spine and trauma surgery with additional information through imaging. The Johns Hopkins Bayview Medical Center in Maryland describes that their intra-operative CT usage has a positive impact on patient outcomes by improving safety, decreasing infections and lowering the risks of complications.[24]
	
	Imaging techniques with a MRT
	Magnetic resonance imaging is used in Neurosurgery:
	
	Before surgery to enable precise planning
	During surgery to support decision making and accounting for brain shift
	After surgery to evaluate the outcome
	An MRT system usually requires a lot of space both in the room and around the patient. It is not possible to perform surgery in a regular MRT room. Thus for step 2, there are two solutions of how to use an MR interoperatively, one is a moveable MRT scanner that can be brought in only when imaging is needed, the other is to transport the patient to an MR scanner in an adjacent room during surgery."
	
	"The latter is the system of choice if pediatric cardiologists, electrophysiologists or neurointerventionalists are major users of the room. "
	
http://www.innovations-report.de/html/berichte/medizintechnik/operieren-im-op-der-zukunft.html

http://www.innovations-report.de/html/berichte/medizin-gesundheit/tickende-bombe-bauch-aortenaneurysma-hybrid-op-213189.html 30.04.18 17:30
	"Der Riss einer erweiterten Bauchschlagader, eines sogenannten Bauchaortenaneurysmas (AAA), birgt tödliche Gefahr: Etwa die Hälfte dieser Patienten erreicht das Krankenhaus nicht mehr lebend.
	Aber auch eine vorbeugend durchgeführte Operation ist nicht ohne Risiko. Doch in sogenannten Hybrid-OP-Sälen lassen sich gefäßchirurgische Patienten jetzt noch sicherer behandeln. In Kombination mit individuell angepassten Gefäßprothesen und weiterentwickelter Kathetertechnik hat sich das Spektrum der Behandlungsmöglichkeiten nun stark erweitert. [..]
	Die klassische Operation über einen großen Schnitt in der Bauchdecke weist nach aktueller Studienlage das beste Langzeitergebnis auf. Hierbei überbrückt der Gefäßchirurg die Schwachstelle in der Schlagader, indem er eine Kunststoffprothese einnäht.
	„Der Eingriff ist für die Patienten jedoch belastend, die Erholungszeit ist lang“, erläutert Debus. Deshalb komme sie nur für etwa 30 bis 40 Prozent der Betroffenen in Frage. „Denn in der AAA-Chirurgie haben wir es überwiegend mit systemerkrankten Risikopatienten zu tun“, so der Gefäßchirurg. Sie sind älter und leiden häufig unter Herz-Kreislauf-Erkrankungen, Fettstoffwechselstörungen und Diabetes. „Bei diesen 60 bis 70 Prozent empfehlen wir als schonende Alternative die endovaskuläre Kathetertechnik.“
	[...]
	Hierbei schiebt der Gefäßchirurg das Aortenimplantat über einen kleinen Zugang in der Leiste in die Blutbahn. Unter Röntgenkontrolle platziert er es in der Schlagader. Dabei muss er anatomische Hindernisse überwinden: „Man kann sich die Bauchschlagader wie ein verkalktes, gewundenes Wasserrohr vorstellen“, erläutert Debus. Zudem zweigten lebenswichtige Gefäße für Darm, Nieren, Leber und Rückenmark von ihr ab, die unbedingt geschont werden müssten. Doch dank sogenannter Hybrid-OPs könnten Gefäßchirurgen heute selbst schwierigste endovaskuläre Aorten-Eingriffe schonend und mit guten Erfolgsaussichten vornehmen.
	Dieser mit hochauflösender Röntgentechnik ausgestattete moderne Operationssaal erlaubt es Ärzten gleichzeitig zu operieren und zu durchleuchten. Treten Komplikationen auf, können Gefäßchirurgen sofort das Vorgehen ändern und die Bauchdecke eröffnen. Zudem lassen sich Gefäßprothesen der neuesten Generation individuell anpassen, das Material ist gleitfähig, biegsam und noch haltbarer."
	



