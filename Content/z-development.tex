\chapter{\iflanguage{ngerman}{Entwicklung des Hybriden OP-Saals}{Development}}
\label{sec:overview}


\subsection{Einordnung des Reifegrads}
1. Der digitale Operationssaal- Kapitel 1
	"In diesem Beitrag werden ein Zeitplan für die Entwicklung
	des digitalen Operationssaals (Digital Operating Room – DOR) über einen Zeitraum
	von etwa 25 Jahren (Abb. 1.1) und die politischen, wirtschaftlichen und industriellen
	Probleme, die sich hieraus ergeben können, ansatzweise vorgestellt."
	
	"In der Technologieentwicklung für den DOR werden vier Bereiche bzw. Dimensionen
	definiert:
	–– Geräte, z. B. Signalerfassung und -aufzeichnung, Robotik, Leit- bzw. Navigationssysteme,
	Simulationstechnologien,
	–– IT-Infrastruktur, z. B. Therapy Imaging and Model Management System (TIMMS),
	Infrastruktur für die Speicherung, Integration, Verarbeitung und Übermittlung
	von patientenspezifischen Daten einschließlich Standardentwicklungen, z. B. zu
	DICOM, IHE und EMR,
	–– Funktionalitäten, einschließlich spezifischer interventioneller Verfahren, patientenspezifische
	Modellierung, Optimierung von chirurgischen Workflows, TIMMSEngines
	und
	–– Visualisierung, einschließlich der Verarbeitung, Übertragung, Anzeige und Speicherung
	von Röntgenbildern, Video- und physiologischen Signalen etc."
	
	"Fünf Phasen/Reifegrade in der Entwicklung des DORs sind hier für das erste Quartal
	des 21. Jahrhunderts definiert:
		2005 +: Maturity Level 1: Die erste Phase der Entwicklung (Reifegrad 1) ist eine
	durch die Industrie geprägte Integration von Technologien. Das kritische Merkmal
	dieser Phase ist die Entwicklung von Systemen zur integrierten Gerätekontrolle. Zu
	den weiteren Technologien in dieser Phase gehören z. B. auch HD-Video, digitale Bildaufnahme
	und -verarbeitung, Boom-Mounted Devices und automatische Befunderstellung.
		2010 +: Maturity Level 2: Die zweite Phase der Entwicklung (Reifegrad 2) kann
	durch die perioperative Prozessoptimierung charakterisiert werden. Als zwei entscheidende
	Merkmale dieser Phase gelten die Entwicklung der präoperativen Bildintegration
	und der navigierten Kontrolle. Zusätzliche Technologien umfassen die
	ersten Erweiterungen zu DICOM in der Chirurgie, die intraoperative Bildaufnahme,
	Modellierung und Simulation sowie intelligente Kameras.
		2015 +: Maturity Level 3: Die dritte Phase der Entwicklung (Reifegrad 3) kann
	durch intraoperative Prozessoptimierung charakterisiert werden. Als zwei entscheidende
	Merkmale dieser Phase gelten die Entwicklung und Anwendung von Workflow-
	Management-Engines und ein umfangreiches DICOM in der Chirurgie. Zusätzliche
	Technologien umfassen den DOR-Prozess-Redesign mit Electronic Medical Record
	(EMR) und Signalintegration, grundlegende IHE-Integrationsprofile für die Chirurgie,
	Smart Walls einschließlich n-dimensionaler Visualisierung und erste Ansätze zu
	einer modellgestützten Intervention.
		2020 +: Maturity Level 4: Die vierte Phase der Entwicklung (Reifegrad 4) kann
	durch herstellerunabhängige Integration von Technologien charakterisiert werden.
	Als weitere kritische Merkmale dieser Phase gelten die Entwicklung der krankenhausbzw.
	unternehmensweiten Interoperabilität und die Anwendung von integrierten
	patientenspezifischen Modellen. Zusätzliche Technologien umfassen das Wissensund
	Entscheidungsmanagement, die quantitative und statistische klinische Bewertung
	sowie IHE-Integrationsprofile für die Chirurgie und Pathologie.
		2025 +: Maturity Level 5: Die fünfte Phase der Entwicklung (Reifegrad 5) kann
	durch intelligente Infrastrukturen und Prozesse charakterisiert werden. Weitere kritische
	Merkmale dieser Phase könnten die Entwicklung von chirurgischen Cockpit-
	Systemen und die weitgehende Umsetzung von integrierten DOR-Architekturen wie
	der TIMMS-Architektur sein. Zusätzliche Technologien umfassen modellbasierte
	medizinische Evidenz (MBME), Zugang zu Peer-to-Peer-chirurgischen Workflow-
	Repositories in Echtzeit, intelligentes Echtzeit-Data-Mining, volle Sprach- und Gestensteuerung,
	Integration von computerassistierter Diagnose (CAD) in Echtzeit sowie
	intelligente (situationsbewusste) Roboter und Geräte."
	
	"Aktuelle Verfahren der Bewertung fokussieren sich auf spezifische Aspekte, wie
	beispielsweise integrierte Gerätesteuerung (Reifegrad 1) oder perioperative Prozessoptimierung
	(Reifegrad 2)."
	
	
\subsection{Zukünftige Entwicklung}
1. Der digitale Operationssaal- Kapitel 2
	"Im Rahmen eines Projektes soll mit dem „Surgical Deck“ ein Prototyp für eine neue
	Generation eines OP-Saals konzipiert und umgesetzt werden. Dabei sollen vor allem
	eine neue Stufe der Integration von Funktionalitäten und ein neuartiges Verständnis
	eines hoch entwickelten Arbeitsplatzes erreicht werden. Das Surgical Deck soll folgende
	Anforderungen erfüllen:
	–– IP-basierte echtzeitfähige Datenstruktur und Interoperabilität: Alle relevanten
	Daten sollen einem digitalen Datenbus in Echtzeit zur Verfügung stehen. Dadurch
	soll eine Möglichkeit der universellen Weiterverarbeitung dieser Daten geschaffen
	werden. Die digital verfügbaren Daten sollen durch Softwarekomponenten,
	sogenannte Middleware, neuartige Funktionen realisieren. Diese Datenverarbeitung
	muss die Anforderungen an die jeweilige Risikoanalyse erfüllen.
	–– Standardisierte Systematik von Funktionen, Arbeitsbereichen, Geräten und Systemen:
	Dadurch soll trotz zunehmender Komplexität die Bedienbarkeit und Übersichtlichkeit
	der Bedien- und Informationssysteme verbessert und eine Standardisierung
	der zunehmend komplexen Prozessschritte im OP erleichtert werden.
	–– Nachweis der klinischen Einsatzfähigkeit: Das Surgical Deck soll die Durchführung
	HNO-chirurgischer Prozeduren ohne Nachteile gegenüber bisherigen OPSystemen
	erlauben. Ausgewählte Parameter sollen Vorteile im Bereich der Ergonomie
	und Betriebssicherheit zeigen."
	HIER NOCHMAL IM BUCH WEITERLESEN; BIN MIR NICHT SICHER OB DAS WIRKLICH PASST; IST EHER HEUTE ALS ZUKUNFT
	
	
	



