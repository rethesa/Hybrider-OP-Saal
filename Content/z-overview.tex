\chapter{\iflanguage{ngerman}{Übersicht Hybrider OP-Saal}{Overview}}
\label{sec:overview}

Weniger invasive Eingriffe, geringeres Risiko und kürzere Operations-und Krankenhausaufenthalte sind Anforderungen die heutzutage an Ärzte und Operationssäle gestellt werden. Anforderungen, welchen ein konventioneller OP-Saal nicht mehr gerecht werden kann, weshalb ein Umstieg zum Hybriden/ Digitalen/ Multifunktionalen oder auch Hochpräzisions genannten OP-Saal statt findet. Egal welche der genannten Bezeichnungen verwendet wird, bei allen Variationen geht es darum den OP-Saal zu digitalisieren und Bildgebende Verfahren wie Röntgen oder Ultraschall während der Operation einzusetzen.

In den folgenden Kapiteln wird immer vom Hybriden OP-Saal die Rede sein und es werden die Unterschiede vom Hybriden zum Standard OP-Saal herausgearbeitet, sowie in welchen Einsatzgebieten der Hybride OP-Saal bereits verwendet wird und welche Vor-und Nachteile dieser mit sich bringt.


\subsection{Hybrider OP-Saal vs. Konventioneller OP-Saal} 

Ein Hybrider OP-Saal ist die Verbindung aus einem sterilen konventionellen OP-Saal mit qualitativ hochwertiger Bildgebung, einem multifunktionalen OP-Tisch, einfacher Datenregistrierung und Dokumentation, ungehindertem Datenaustausch innerhalb und außerhalb des OPs und der Steuerung aller vorhandenen Geräte und Systeme über ein einzelnes Interface \cite{HybriderVsKonventioneller}, \cite{KarlStorz}. 

Es wird nicht zwischen unterschiedlichen Abteilungen diskriminiert, sondern ein Hybrider OP-Saal ist ein chirurgischer Arbeitsbereich, welcher fachbereichsübergreifend von der Neurochirurgie, Gefäßchirurgie, Onkologie, Kardiologie, Unfallchirurgie und vielen weiteren Bereichen genutzt werden kann \cite{Getinge}.
Mobile C-Bogen sowie Ultraschall- und Endoskopiegeräte trifft man in den meisten konventionellen Operationssälen an \cite{TechnicalConsiderations}, für komplexe Operationsvorgänge wie bspw. bei Transkathetertechniken und für die Visualisierung der dünnen Führungsdrähte, wird jedoch eine leistungsstärkere Ausrüstung benötigt. Die mobilen C-Bogen werden durch befestigte ersetzt und zur ergänzenden Ausrüstung des Hybriden OP-Saals gehören noch Geräte für die intraoperative Bildgebung, wie Computertomographen (CT), Magnetresonanztomographen (MRT) oder Angiografieanlagen. Diese ermöglichen es, während eines Operationsvorgangs Diagnose und Therapiekontrolle vorzunehmen, sowie minimal invasive Behandlungsverfahren durchzuführen \cite{SHG-Kliniken}. Auch ist es möglich "während der Operation - und nicht erst danach - Bildkontrollen durch[zu]führen und allenfalls Korrekturmaßnahmen [zu] ergreifen" \cite{OPderZukunft}.

TODO:
intraoperative Aniographie --> Darstellung von Gefäßen (Blutgefäßen) mittels Röntgen oder MRT
Fluroskopie, 3D Imaging, DSA,...

\subsection{Vorteile und wichtige Einsatzgebiete}

Der Hybride OP Saal verspricht präziseres, sichereres und schonenderes Operieren \cite{DresdnerUniklinikum}


\textbf{Neurochirurgie:}
neurochirurgie --> brain shift

\textbf{Gefäßchirurgie:}
Die Gefäßchirurgie profitiert besonders von Hybriden OP-Sälen, da die intraoperative Echtzeitbildgebung erlaubt, den Fortschritt eines Katheters durch die Gefäße zu beobachten und das Ergebnis, wie das Setzen und richtige Sitzen einer Prothese, zu kontrollieren.
Nehmen wir als Beispiel ein Aortenaneurysma, also eine krankhafte Erweiterung der Hauptschlagader. Diese Aneurysmen können ruptieren und zu lebensbedrohlichen Blutungen führen, um das zu verhindern muss ab einem gewissen Durchmesser der Erweiterung eine Prothese gesetzt werden. Die Operation kann entweder als offene Operation mit konventionellen Prothesen durchgeführt werden oder mit einem schonenden Verfahren und Implantation großer maßgefertigter Gefäßprothesen im Hybrid OP-Saal. Die offene Operation, bei der bei Bauchaneurysmen ein großer Schnitt in die Bauchdecke gemacht werden muss, weist zwar gute Langzeitergebnisse auf aber ist sehr belastet und mit langen Erholungszeiten verbunden. Da Aortenaneurysmen verstärkt im höheren Alter, also gegen 60 Jahre und Älter auftreten, kommt für viele Patienten eine offene Operation nicht mehr in Frage, da die Belastungen denen sie dabei ausgesetzt würden zu groß sind. 
Beim schonenden Verfahren wird vor der Operation ein CT Bild angefertigt, welches dann mit den während der Operation entstehenden zwei- oder dreidimensionalen Bildern der Röntgenkontrolle zusammengeführt wird. Dies ermöglicht, dass die Prothese über einen kleinen Zugang in der Leiste mithilfe eines Katheters minimalinvasiv durch das virtuell abgebildete Gefäß navigiert und gesetzt werden kann. Eine zusätzliche Software zur Navigationshilfe trägt dazu bei, die Prothese sicher und schnell durch die Gefäße ans Ziel zu bringen \cite{Aortenaneurysma}, \cite{DresdnerUniklinikum}, \cite{TickendeBombeImBauch}.

Zusammenfassend lässt sich also sagen, dass durch den Einsatz des Hybrid OPs die Belastungen, denen Patienten durch den Eingriff ausgesetzt sind, verringert werden was wiederum zu kürzeren Krankenhausaufenthalten und zu Kosteneinsparungen in der Nachbetreuung und -behandlung führt. 


TODO: Ältere Menschen können noch behandelt werden, keine großen Narben, weniger Belastend; geringe Strahlungsbelastung von neuen Angiographieanlagen 
 
\textbf{Onkologie:}
In der Onkologie hat man den großen Vorteil, dass nachdem ein Tumor entfernt aber noch bevor die Operation abgeschlossen wurde, ein Kernspin gemacht werden kann ohne den Patienten transportieren zu müssen. So kann man sicherstellen, dass tatsächlich keine Rückstände des Tumors übersehen wurden bzw. sollte dem nicht der Fall sein, dann kann direkt Nachkorrigiert werden. Durch diese Korrekturmöglichkeit können manche Folgeoperationen dem Patienten erspart bleiben \cite{AerzteZeitung}.

\textbf{Kardiologie:}
cardiac -->kariologie, bsp: herzchirurgie

\textbf{Unfallchirurgie:}
Unfallchirurgie -->

VORTEILE:
- Reduktion der Komplexität, Bypass Zeit und Risiko  sowie bessere Operationsergebnisse
- Qualitätskontrollen währen Eingriff --> ermöglichen Korrekturmöglichkeiten
- mehr Sicherheit
- Patienten können optimal positioniert werden und somit optimale Röntgenbilder
- präziseres, sicheres und schonenderes operieren
- weniger traumatische Erfahrung für Patienten, Minimierung der benötigten Schnitte mit gleichem Ergebnis
- präzises Planen vor der Operation, Unterstützung zum Treffen von Entscheidungen während der OP, nach Operation um das Ergebnis zu beurteilen oder Nachzukorrigieren
- Ältere Patienten können noch behandelt werden
- Erholzeit wird verkürzt

- minimalinvasiv --> weniger belastend und kleinere Narben, sowie kürzere Krankenhausaufenthalte

\subsection{Nachteile}

STRAHLENBELASTUNG FÜR PERSONAL
%%WIKIPEDIA: https://en.wikipedia.org/wiki/Hybrid_operating_room
"X-ray radiation is ionizing radiation, thus exposure is potentially harmful. Compared to a mobile C-Arm, which is classically used in surgery, CT scanners and fixed C-Arms work on a much higher energy level, which induces higher dose. Therefore, it is very important to monitor radiation dose applied in a hybrid OR both for the patient and the medical staff.[33]"

1. http://www.innovations-report.de/html/berichte/medizintechnik/dresdner-uniklinikum-nimmt-high-end-hybrid-op-in-betrieb.html	 30.04.18 16:00
"Das strahlungsarme High-Tech-Gerät kommt den Patienten ebenso zugute wie den Chirurgen, denn die in der Regel mit Kathetern vorgenommenen Eingriffe finden unter laufender Röntgenkontrolle statt. Die hochmoderne Anlage liefert trotz niedriger Dosis und der verringerten Gabe von Kontrastmitteln höher aufgelöste Bilder."


5. Der digitale Operationssaal- Kapitel 3
"Die Austauschbarkeit
von einzelnen Medizinprodukten ist beschränkt. Eine Integration von Drittanbieterkomponenten
kann nur in Kooperation mit dem Hersteller des Integrationssystems
erfolgen [5]. Als wesentlicher Grund für den Einsatz von monolithischen Lösungen
werden von Herstellern die Notwendigkeit der Konformitätsbewertung von Medizinprodukten
sowie die in diesem Zusammenhang entstehende Problematik des Risikomanagements
vernetzter Medizinprodukte angeführt."



	



	
