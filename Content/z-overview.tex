\chapter{\iflanguage{ngerman}{Übersicht Hybrider OP-Saal}{Overview}}
\label{sec:overview}

Weniger invasive Eingriffe, geringeres Risiko und kürzere Operations-und Krankenhausaufenthalte sind Anforderungen die heutzutage an Ärzte und Operationssäle gestellt werden \cite{DerDigitaleOperationssaal}. Anforderungen, welchen ein konventioneller OP-Saal nicht mehr gerecht werden kann, weshalb ein Umstieg zum Hybriden/ Digitalen/ Multifunktionalen oder auch Hochpräzisions genannten OP-Saal statt findet. Egal welche der genannten Bezeichnungen verwendet wird, bei allen Variationen geht es darum den OP-Saal zu digitalisieren und Bildgebende Verfahren wie Röntgen oder Ultraschall während der Operation einzusetzen.

In den folgenden Kapiteln wird immer vom Hybriden OP-Saal die Rede sein und es werden die Unterschiede vom Hybriden zum Standard OP-Saal herausgearbeitet, sowie in welchen Einsatzgebieten der Hybride OP-Saal bereits verwendet wird und welche Vor-und Nachteile dieser mit sich bringt.

\subsection{Hybrider OP-Saal vs. Konventioneller OP-Saal} 

Ein Hybrider OP-Saal ist die Verbindung aus einem sterilen konventionellen OP-Saal mit qualitativ hochwertiger Bildgebung, einem multifunktionalen OP-Tisch, einfacher Datenregistrierung und Dokumentation, ungehindertem Datenaustausch innerhalb und außerhalb des OPs und der Steuerung aller vorhandenen Geräte und Systeme über ein einzelnes Interface \cite{HybriderVsKonventioneller,KarlStorz}. Es wird nicht zwischen unterschiedlichen Abteilungen diskriminiert, sondern ein Hybrider OP-Saal ist ein chirurgischer Arbeitsbereich, welcher fachbereichsübergreifend von der Neurochirurgie, Gefäßchirurgie, Onkologie, Kardiologie, Unfallchirurgie und vielen weiteren Bereichen genutzt werden kann \cite{Getinge}.\\
Mobile C-Bogen sowie Ultraschall- und Endoskopiegeräte trifft man in den meisten konventionellen Operationssälen an \cite{TechnicalConsiderations}, für komplexe Operationsvorgänge wie bspw. bei Transkathetertechniken und für die Visualisierung der dünnen Führungsdrähte, wird jedoch eine leistungsstärkere Ausrüstung benötigt. Die mobilen C-Bogen werden durch befestigte ersetzt und zur ergänzenden Ausrüstung des Hybriden OP-Saals gehören noch Geräte für die intraoperative Bildgebung, wie Computertomographen (CT), Magnetresonanztomographen (MRT) oder Angiografieanlagen. Diese ermöglichen es, während eines Operationsvorgangs Diagnose und Therapiekontrolle vorzunehmen, sowie minimal invasive Behandlungsverfahren durchzuführen \cite{SHG-Kliniken}. Auch ist es möglich "während der Operation - und nicht erst danach - Bildkontrollen durch[zu]führen und allenfalls Korrekturmaßnahmen [zu] ergreifen" \cite{OPderZukunft}.

%TODO:
%intraoperative Aniographie --> Darstellung von Gefäßen (Blutgefäßen) mittels Röntgen oder MRT
%Fluroskopie, 3D Imaging, DSA,...

%The primary components of the hybrid suite include
%intraoperative angiography and fluoroscopy, as well as
%carefully designed operating tables to accommodate
%and optimize the usefulness of the radiographic equipment. \cite{ORofTheFuture}

%TODO

\subsection{Vorteile und wichtige Einsatzgebiete}

Wie genau Hybride OP-Säle Operationsvorgänge unterstützen, soll  an dieser Stelle in der Neurochirurgie anhand der Kompensation von Brain Shift, in der Gefäßchirurgie anhand des Setzens von Prothesen bei Aortenaneurysmen und in der Onkologie anhand der Tumorentfernung genauer erklärt werden.

\textbf{Neurochirurgie:}
Das Gehirn ist eine sehr komplexe dreidimensionale Struktur, bei der zusätzlich sogenannte Brain Shifts (Verschiebungen der Gehirnstruktur) während einer Operation auftreten können. Ursachen für dieses Phänomen können die Entfernung oder das Anschwellen von Gewebe, sowie der Verlust von Hirnwasser sein. 
Kommt es während einer Operation zu den oben genannten Verschiebungen, dann stimmen die vor einer Operation angefertigten CT oder MR Bilder der Gehirnstruktur nicht mehr mit der aktuellen Struktur überein. Bild gesteuerte neurochirurgische Systeme (IGNS), die normalerweise während der Operation als Navigationshilfe dienen, können dann nur noch begrenzt richtig arbeiten. Durch die inkorrekten Bilddaten entsteht ein größeres Risiko für den Patienten und die Operationsergebnisse werden verschlechtert.\\
Intraoperativer Ultraschall (iUS) oder intraoperative MR-Bilder (iMR) können diesem Problem entgegen wirken und die fehlenden Informationen ergänzen, sodass das IGN System wieder sinnvoll verwendbar ist.
Der Einsatz von iMR ermöglicht, während der Operation regelmäßig hochauflösende Bilder von der Gewebestruktur des Gehirns anzufertigen und im Falle eines Brain Shifts entsprechend darauf zu reagieren. Auch bei anderen neurochirurgischen Anwendungen wie der Hirnbiopsie, Entfernung von Tumoren oder der Drainage von Zysten kann iMR von großem Vorteil sein.\\
%In addition, the high-field system also benefits from functional techniques including MR spectroscopy, 
%functional MRI, MR angiography, chemical shift imaging, and diffusion-weighted imaging
Obwohl iMR als verlässlichste Option gilt um mit Brain Shift umzugehen, hat iUS den großen Vorteil sehr kostengünstig Echtzeitzeitbilder zu produzieren. Die Kombination aus einem präoperativen MR Bild und intraoperativen Ultraschall reicht aus, um Gewebeveränderungen zu registrieren und die Korrektheit des IGN Systems zu beurteilen. Es wird mittlerweile eine Genauigkeit von ungefähr 1,36mm erreicht und im Gegensatz zu iMR mit einer Bildaufnahmezeit von 15 Minuten kann ein iUS Bild in nur 5 Minuten aufgenommen werden.
Aufgrund der schlechten Bildqualität wird Ultraschall bisher nur sehr begrenzt in der Neurochirurgie verwendet. Mit Fortschreiten der derzeitigen Entwicklung könnte sich das jedoch bald ändern. Genauer wird darauf im Kapitel \glqq Neue Technologien\grqq{}  eingegangen. \cite{BrainShiftInTumorResection}
%TODO LRS und Stereo Vision; und ich glaube auf US usw. muss ich doch hier schon genauer eingehen, dafür ist die Technologie schon fast zu weit fortgeschritten
%- Krankenhausaufenthalt mit iMR von gut 9 auf knapp 5 Tage verkürzt bei Tumorentfernung \cite{BrainShiftInTumorResection}

\textbf{Gefäßchirurgie:}
Die Gefäßchirurgie profitiert besonders von Hybriden OP-Sälen, da die intraoperative Echtzeitbildgebung erlaubt, den Fortschritt eines Katheters durch die Gefäße zu beobachten und das Ergebnis, wie das Setzen und richtige Sitzen einer Prothese, zu kontrollieren.\\
Nehmen wir als Beispiel ein Aortenaneurysma, also eine krankhafte Erweiterung der Hauptschlagader. Diese Aneurysmen können ruptieren und zu lebensbedrohlichen Blutungen führen, um das zu verhindern muss ab einem gewissen Durchmesser der Erweiterung eine Prothese gesetzt werden. Die Operation kann entweder als offene Operation mit konventionellen Prothesen durchgeführt werden oder mit einem schonenden Verfahren und Implantation großer maßgefertigter Gefäßprothesen im Hybrid OP-Saal. Die offene Operation, bei der bei Bauchaneurysmen ein großer Schnitt in die Bauchdecke gemacht werden muss, weist zwar gute Langzeitergebnisse auf aber ist sehr belastet und mit langen Erholungszeiten verbunden. Da Aortenaneurysmen verstärkt im höheren Alter, also gegen 60 Jahre und Älter auftreten, kommt für viele Patienten eine offene Operation nicht mehr in Frage, da die Belastungen denen sie dabei ausgesetzt würden zu groß sind. \\
Beim schonenden Verfahren wird vor der Operation ein CT Bild angefertigt, welches dann mit den während der Operation entstehenden zwei- oder dreidimensionalen Bildern der Röntgenkontrolle zusammengeführt wird. Dies ermöglicht, dass die Prothese über einen kleinen Zugang in der Leiste mithilfe eines Katheters minimalinvasiv durch das virtuell abgebildete Gefäß navigiert und gesetzt werden kann. Eine zusätzliche Software zur Navigationshilfe trägt dazu bei, die Prothese sicher und schnell durch die Gefäße ans Ziel zu bringen \cite{Aortenaneurysma,DresdnerUniklinikum,TickendeBombeImBauch}.
%TODO Fluroskopie??? --> also laufende Röntgenkontrolle

\textbf{Onkologie:}
In der Onkologie hat man den großen Vorteil, dass nachdem ein Tumor entfernt und bevor die Operation abgeschlossen wurde, ein Kernspin mit dem MRT gemacht werden kann ohne den Patienten transportieren zu müssen. So kann man sicherstellen, dass tatsächlich keine Rückstände des Tumors übersehen wurden bzw. sollte dem nicht der Fall sein, dann kann direkt Nachkorrigiert werden. Durch diese Korrekturmöglichkeit können manche Folgeoperationen dem Patienten erspart bleiben \cite{AerzteZeitung}.\\
Die Zweckmäßigkeit des intraoperativen MRT wurde damit bestätigt, dass in mehr als einem drittel der Fälle, in denen eine vollständige Entfernung des Tumors angenommen wurde, Rückstände festgestellt und somit nachkorrigiert werden musste \cite{BrainShiftInTumorResection}.

%\textbf{Kardiologie:}

%TODO: Ältere Menschen können noch behandelt werden, keine großen Narben, weniger Belastend; geringe %Strahlungsbelastung von neuen Angiographieanlagen 

Zusammenfassen lässt sich also sagen, dass Hybride OP-Säle präziseres, sicheres und schonenderes operieren \cite{DresdnerUniklinikum} durch minimalinvasive Eingriffe ermöglichen, welche weniger belastend für den Patienten sind und kleinere Narben hinterlassen. Das wiederum  führt zu kürzeren Krankenhausaufenthalten und zu Kosteneinsparungen in der Nachbetreuung und -behandlung. \\
Der beliebig positionierbare OP-Tisch, zusammen mit dem hochbeweglichen C-Bogen  ermöglichen optimale Röntgenbildaufnahmen \cite{DresdnerUniklinikum}, die zur Unterstützung vor, während und nach der Operation eingesetzt werden können. Entscheidungen über den Fortlauf der Operation können so besser getroffen werden, Ergebnisse beurteilt und gegebenenfalls nachkorrigiert werden.

\subsection{Nachteile}

Der Hybride OP-Saal weist sehr viele positive Aspekte auf, doch sind die negativen nicht zu vernachlässigen und werden deshalb im Folgenden besprochen.

Der Hybride OP-Saal wird in erster Linie durch die bildgebenden Verfahren definiert, doch wie bereits aus der Radiologie bekannt ist, müssen immer auch mögliche "Mess-, Rekonstruktions- und Modellierungsfehler berücksichtigt werden" \cite{DerDigitaleOperationssaal}. 
So können Partialvolumenartefakte Grund dafür sein, dass Tumore in der falschen Größe dargestellt werden oder bei CT Bildern die besonders kleinen Läsionen (< 1cm) überhaupt nicht in der Bildausgabe erkenntlich sind. Diese möglichen Fehler müssen in der Operationsplanung und späteren Ausführung berücksichtigt werden \cite{DerDigitaleOperationssaal}.

Hinzu kommt, dass Operationen im Hybriden OP-Saal teilweise unter laufender Röntgenkontrolle statt finden. Dieser Belastung ist dann nicht nur der Patient ausgesetzt sondern das gesamte behandelnde Team bekommt Röntgenstrahlung ab. Je nach Abstand, Winkel und Höhe zum Patienten während der Bildkontrolle, liegt die maximale Strahlungsreduktion zwischen 61 und 91\% (bei einem Patienten mit 65kg Körpergewicht). Je nach Größe und Gewicht des Patienten kann es aber auch zu einem höheren Streustrahlenanteil kommen und damit zu einer höheren Belastung für das behandelnde Team.
Nimmt man an, dass pro Operation durchschnittlich zwei 3D Scans gemacht werden, dann summiert sich die Strahlendosis für ein Teammitglied, bei 100 Operationen im Jahr, auf ungefähr 400µSv. Dies entspricht bereits 7\% der jährlichen maximalen Dosis und muss deshalb stark wie möglich verhindert werden \cite{RadiationExposure}.

Ein anderes Problem entsteht beispielsweise bei der Tumorentfernung im Gehirn mit intraoperativem MRT. Um kleine Veränderungen (wie Brain Shift) frühzeitig erkennen und darauf reagieren zu können, sind viele regelmäßige Bildaufnahmen nötigt. Gleichzeitig muss für jede Bildaufnahme die Operation unterbrochen und für jedes Bild ein gewisser Zeitaufwand aufgebracht werden. Für ein iMR-Bild muss etwa mit einer Aufnahmezeit von circa 15 Minuten gerechnet werden \cite{BrainShiftInTumorResection}. Jede Aufnahme führt damit auch zu einer Verlängerung der Operationszeit und es muss somit ein Kompromiss zwischen Zeitaufwand und Bildhäufigkeit gefunden werden.\\
Alternativ besteht in diesem Fall immer noch die Möglichkeit, eine schlechtere Bildqualität in Kauf zu nehmen und dafür nur eine Aufnahmezeit von 5 Minuten, wenn man stattdessen zum US-Gerät greift. Unter diesen Umständen sollte man aber nicht vergessen, dass Ultraschall, im Gegensatz zu Magnetresonanztomografie, nicht kontaktlos verwendet werden kann, was somit immer auch ein höheres Infektionsrisiko birgt \cite{BrainShiftInTumorResection}.

Ein Punkt der wohl vielen Krankenhäusern Kopfzerbrechen bereitet, werden wohl die hohen Investitions- und Wartungskosten eines Hybriden OP-Saals sein. Die benötigte Raumgröße und Ausrüstung führen dazu, dass ein Hybrider OP-Saal in der Anschaffung mehr als doppelt so teuer und in der Wartung fast doppelt so teuer ist wie ein konventioneller OP-Saal \cite{HybridOR}. 
Ob diese Kosten gerechtfertigt sind und lohnend sind wird im nächsten Kapitel in der Kosten-Nutzen Analyse erörtert.

