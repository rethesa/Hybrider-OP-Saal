\chapter{\iflanguage{ngerman}{Übersicht Hybrider OP-Saal}{Overview}}
\label{sec:overview}

\subsection{Hybrider OP-Saal vs. Standard OP-Saal / Stand der Technik}
1. Der digitale Operationssaal- Kapitel 2
	"Die Anforderungen an einen Operationssaal sind in den vergangenen fünf Jahren
	deutlich gestiegen. Das gilt besonders für alle spezialchirurgischen Fächer mit minimal-
	invasiven Zugängen und endoskopischer Visualisierung wie z. B. der HNO-Chirurgie
	(Abb. 2.1). Neue Funktionen wie Instrumentennavigation, Kollisionswarnung,
	Neuromonitoring, intraoperativer Ultraschall, Messsysteme, Mikromanipulatoren,
	Telekonferenzen, kontinuierliche Aufzeichnungen des Eingriffs, Einblick in die Elektronische
	Patientenakte (EPR) und das radiologische Archiv (S-PACS) erweitern die
	Möglichkeiten eines OP-Saals drastisch."
	
	"Die derzeit verfügbaren Operationssäle werden meist in konventionelle und integrierte
	OP-Säle unterteilt (z. B. OR1 der Karl Storz GmbH und Co. KG; Endoalpha der
	Olympus Deutschland GmbH; iSuite der Stryker GmbH und Co. KG) [1–3]. Die Definition
	des „integrierten OP-Saals“ ist dabei jedoch unscharf. Häufig wird bereits die Realisierung
	minimaler Veränderungen wie RGB-Licht in den Wänden oder eine versteckte
	Kabelführung mit diesem Begriff beworben. Diese OP-Säle umfassen vor allem eine
	Verbesserung der architektonischen sowie räumlichen und nur teilweise der funktionellen
	Integration.
	Weiterführende Integrationslösungen verfügen heute über einen proprietären
	Datenbus zur Vernetzung ausgewählter Komponenten [4, 5]. Durch einen solchen
	Bus ist die zentrale Bedienung von Medizingeräten über ein gemeinsames Interface
	möglich. Besonders diese Vorarbeiten sind heute die Voraussetzungen für die nächste
	Generation von OP-Systemen."

"http://www.ctsnet.org/article/cardiovascular-hybrid-or-clinical-technical-considerations"	30.04.18 16:15
	"Mobile C-arms, ultrasound, and endoscopy are standard of care for many operations. However, complex transcatheter techniques demand high powered equipment to visualize thin guide wires, quantify small vessel diameters, and evaluate delicate anastomoses. Because of their size and complexity these integrated endovascular suites or Hybrid ORs require special considerations, planning, and design as well as new skills to be learned by the team."

http://www.innovations-report.de//html/berichte/medizin-gesundheit/saarlaendische-shg-kliniken-setzen-hybrid-op-201121.html 30.04.18 15:45	
	"Ein Hybrid-Operationssaal ist die Kombination aus hochsterilem Operationssaal und Systemen zur intraoperativen Bildgebung wie Computertomografen oder Angiografieanlagen. Dadurch ist es möglich, Diagnostik, interventionelle und/oder operative Therapie und Therapiekontrolle in einer Sitzung vorzunehmen.
	Der Betrieb von Hybrid-Operationssälen ermöglicht weniger invasive Behandlungsverfahren. Gleichzeitig wird das Zusammenarbeiten verschiedener Fachdisziplinen vorangebracht, was zur qualitativen Verbesserung der Behandlungen führt. Die häufigsten Anwendungen finden sich bisher in der Kardiologie und Herzchirurgie (minimalinvasive Implantationen von Herzklappen) sowie in der Gefäßchirurgie."
	
http://www.innovations-report.de/html/berichte/medizintechnik/dresdner-uniklinikum-nimmt-high-end-hybrid-op-in-betrieb.html 30.04.18 16:00
	"Der Begriff „Hybrid-OP“ steht für einen modernen Operationssaal der mit Geräten medizinischer Bildgebung kombiniert wird. In einem solchen Saal lassen sich beispielsweise auch bei offenen OP Röntgenbilder anfertigen und Interventionen mit Kathetern vornehmen."
\subsection{Geräte und System im Hybriden OP-Saal}	
http://www.maria-online.com/home/article.php?lg=de&q=Hybrid-OP
	" Angiografieanlagen, Computertomographen oder Magnetresonanztomographen"
	 "Bildgebung in Form von mobilen C-Bögen, Ultraschall und Endoskopie gehört seit langem zur Standardausstattung im OP."
	 "Die Behandlung von Klappenkrankheiten und die chirurgische Therapie von Rhythmusstörungen und Aortenaneurysmen können von den Bildgebungsmöglichkeiten des Hybrid-OPs profitieren. In diesen Bereichen ist intraoperative (3D)-Bildgebung bereits sehr verbreitet."

WIKIPEDIA: https://en.wikipedia.org/wiki/Hybrid_operating_room
	"The most common imaging modality to be used in hybrid ORs is a C-Arm. Expert consensus rates the performance of mobile C-arms in hybrid ORs as insufficient, because the limited power of the tube impacts image quality, the field of view is smaller for image-intensifier systems than for flat-panel detector systems and the cooling system of mobile C-Arms can lead to overheating after just a few hours, which can be too short for lengthy surgical procedures or for multiple procedures in a row, that would be needed to recover the investment in such a room."
	
	"surgical lights or booms"
	
	"OR table
	The selection of the OR table depends on the primary use of the system. Interventional tables with floating table tops and tilt and cradle compete with fully integrated flexible OR tables. Identification of the right table is a compromise between interventional and surgical requirements.[1][29] Surgical and interventional requirements may be mutually exclusive. Surgeons, especially orthopedic, general and neurosurgeons usually expect a table with a segmented tabletop for flexible patient positioning. For imaging purposes, a radiolucent tabletop, allowing full body coverage, is required. Therefore, non-breakable carbon fibre tabletops are used."
	
http://www.innovations-report.de/html/berichte/medizintechnik/operieren-im-op-der-zukunft.html 30.04.18 16:00	
	"Ab Mitte Februar 2017 stehen im Intensivbehandlungs-, Notfall- und Operationszentrum (INO) im Inselspital allen operativen Fachgebieten drei neue OP-Säle mit integrierter Computertomografie (CT) und Magnetresonanztomografie (MRT) zur Verfügung. Zusammen mit dem Hybrid-OP, der die intraoperative Angiografie erlaubt, bilden sie einen in der Schweiz einzigartigen OP-Bereich."

http://www.ctsnet.org/article/cardiovascular-hybrid-or-clinical-technical-considerations 30.04.18 17:00	
	"Mobile C-arms have been commonly used in cardiac surgery and they are readily available in every department, e.g. for pacemaker implantation. Mobile C-arms may depict larger stents or catheters well. However, their technical specifications do not meet the recommendations of the cardiology societies [...] 
	Mobile C-arms generally have a heat storage capacity of up to 300,000 heat units (HU) (exception: rare water cooled systems). A heat storage capacity of more than 1 million HU is recommended by cardiology societies for cathlabs to avoid overheating and a dangerous shut down during complex procedures which may occur in mobile C-arms. For these reasons expert consensus recommends use of fixed C-arms (8). A semi-mobile system with a fixed generator (80 kW, AXIOM Artis U; Siemens AG, Forchheim, Germany) may accommodate high-power imaging demands even in average sized operating rooms too small to house a fixed C-arm (<45 m2)."
	
	"Fluoroscopy and acquisition are the basic and most important imaging modes and offered by all systems. Since fluoroscopy needs much less radiation dose, brilliant fluoroscopy images are the predominantly used images during the procedure. However, modern angiography systems offer advanced imaging and post-processing capabilities including image fusion with any type of previously acquired 3D volumes (e.g. CT, MR, PET, SPECT images), guidance, or 3D imaging.  "
	
	
\subsection{Einordnung des Reifegrads}
1. Der digitale Operationssaal- Kapitel 1
	"In diesem Beitrag werden ein Zeitplan für die Entwicklung
	des digitalen Operationssaals (Digital Operating Room – DOR) über einen Zeitraum
	von etwa 25 Jahren (Abb. 1.1) und die politischen, wirtschaftlichen und industriellen
	Probleme, die sich hieraus ergeben können, ansatzweise vorgestellt."

	"In der Technologieentwicklung für den DOR werden vier Bereiche bzw. Dimensionen
	definiert:
	–– Geräte, z. B. Signalerfassung und -aufzeichnung, Robotik, Leit- bzw. Navigationssysteme,
	Simulationstechnologien,
	–– IT-Infrastruktur, z. B. Therapy Imaging and Model Management System (TIMMS),
	Infrastruktur für die Speicherung, Integration, Verarbeitung und Übermittlung
	von patientenspezifischen Daten einschließlich Standardentwicklungen, z. B. zu
	DICOM, IHE und EMR,
	–– Funktionalitäten, einschließlich spezifischer interventioneller Verfahren, patientenspezifische
	Modellierung, Optimierung von chirurgischen Workflows, TIMMSEngines
	und
	–– Visualisierung, einschließlich der Verarbeitung, Übertragung, Anzeige und Speicherung
	von Röntgenbildern, Video- und physiologischen Signalen etc."

	"Fünf Phasen/Reifegrade in der Entwicklung des DORs sind hier für das erste Quartal
	des 21. Jahrhunderts definiert:
		2005 +: Maturity Level 1: Die erste Phase der Entwicklung (Reifegrad 1) ist eine
	durch die Industrie geprägte Integration von Technologien. Das kritische Merkmal
	dieser Phase ist die Entwicklung von Systemen zur integrierten Gerätekontrolle. Zu
	den weiteren Technologien in dieser Phase gehören z. B. auch HD-Video, digitale Bildaufnahme
	und -verarbeitung, Boom-Mounted Devices und automatische Befunderstellung.
		2010 +: Maturity Level 2: Die zweite Phase der Entwicklung (Reifegrad 2) kann
	durch die perioperative Prozessoptimierung charakterisiert werden. Als zwei entscheidende
	Merkmale dieser Phase gelten die Entwicklung der präoperativen Bildintegration
	und der navigierten Kontrolle. Zusätzliche Technologien umfassen die
	ersten Erweiterungen zu DICOM in der Chirurgie, die intraoperative Bildaufnahme,
	Modellierung und Simulation sowie intelligente Kameras.
		2015 +: Maturity Level 3: Die dritte Phase der Entwicklung (Reifegrad 3) kann
	durch intraoperative Prozessoptimierung charakterisiert werden. Als zwei entscheidende
	Merkmale dieser Phase gelten die Entwicklung und Anwendung von Workflow-
	Management-Engines und ein umfangreiches DICOM in der Chirurgie. Zusätzliche
	Technologien umfassen den DOR-Prozess-Redesign mit Electronic Medical Record
	(EMR) und Signalintegration, grundlegende IHE-Integrationsprofile für die Chirurgie,
	Smart Walls einschließlich n-dimensionaler Visualisierung und erste Ansätze zu
	einer modellgestützten Intervention.
		2020 +: Maturity Level 4: Die vierte Phase der Entwicklung (Reifegrad 4) kann
	durch herstellerunabhängige Integration von Technologien charakterisiert werden.
	Als weitere kritische Merkmale dieser Phase gelten die Entwicklung der krankenhausbzw.
	unternehmensweiten Interoperabilität und die Anwendung von integrierten
	patientenspezifischen Modellen. Zusätzliche Technologien umfassen das Wissensund
	Entscheidungsmanagement, die quantitative und statistische klinische Bewertung
	sowie IHE-Integrationsprofile für die Chirurgie und Pathologie.
		2025 +: Maturity Level 5: Die fünfte Phase der Entwicklung (Reifegrad 5) kann
	durch intelligente Infrastrukturen und Prozesse charakterisiert werden. Weitere kritische
	Merkmale dieser Phase könnten die Entwicklung von chirurgischen Cockpit-
	Systemen und die weitgehende Umsetzung von integrierten DOR-Architekturen wie
	der TIMMS-Architektur sein. Zusätzliche Technologien umfassen modellbasierte
	medizinische Evidenz (MBME), Zugang zu Peer-to-Peer-chirurgischen Workflow-
	Repositories in Echtzeit, intelligentes Echtzeit-Data-Mining, volle Sprach- und Gestensteuerung,
	Integration von computerassistierter Diagnose (CAD) in Echtzeit sowie
	intelligente (situationsbewusste) Roboter und Geräte."

	"Aktuelle Verfahren der Bewertung fokussieren sich auf spezifische Aspekte, wie
	beispielsweise integrierte Gerätesteuerung (Reifegrad 1) oder perioperative Prozessoptimierung
	(Reifegrad 2)."


\subsection{Zukünftige Entwicklung}
1. Der digitale Operationssaal- Kapitel 2
	"Im Rahmen eines Projektes soll mit dem „Surgical Deck“ ein Prototyp für eine neue
	Generation eines OP-Saals konzipiert und umgesetzt werden. Dabei sollen vor allem
	eine neue Stufe der Integration von Funktionalitäten und ein neuartiges Verständnis
	eines hoch entwickelten Arbeitsplatzes erreicht werden. Das Surgical Deck soll folgende
	Anforderungen erfüllen:
	–– IP-basierte echtzeitfähige Datenstruktur und Interoperabilität: Alle relevanten
	Daten sollen einem digitalen Datenbus in Echtzeit zur Verfügung stehen. Dadurch
	soll eine Möglichkeit der universellen Weiterverarbeitung dieser Daten geschaffen
	werden. Die digital verfügbaren Daten sollen durch Softwarekomponenten,
	sogenannte Middleware, neuartige Funktionen realisieren. Diese Datenverarbeitung
	muss die Anforderungen an die jeweilige Risikoanalyse erfüllen.
	–– Standardisierte Systematik von Funktionen, Arbeitsbereichen, Geräten und Systemen:
	Dadurch soll trotz zunehmender Komplexität die Bedienbarkeit und Übersichtlichkeit
	der Bedien- und Informationssysteme verbessert und eine Standardisierung
	der zunehmend komplexen Prozessschritte im OP erleichtert werden.
	–– Nachweis der klinischen Einsatzfähigkeit: Das Surgical Deck soll die Durchführung
	HNO-chirurgischer Prozeduren ohne Nachteile gegenüber bisherigen OPSystemen
	erlauben. Ausgewählte Parameter sollen Vorteile im Bereich der Ergonomie
	und Betriebssicherheit zeigen."
	HIER NOCHMAL IM BUCH WEITERLESEN; BIN MIR NICHT SICHER OB DAS WIRKLICH PASST; IST EHER HEUTE ALS ZUKUNFT


	
