\Abstract{
Deutsche Zusammenfassung

-übergeordnete Bedeutung des Themas
-Fragestellung der Arbeit
-wichtigste Thesen
-wissenschaftliche Methode
-Schlussfolgerungen / Ergebnisse

Tipps für das Schreiben eines Abstracts
Formulieren Sie das Abstract in der Gegenwart. Legen Sie dar, was die Arbeit leistet. Es geht nicht darum, was die Arbeit leisten wird.
Wählen Sie einen spannenden Einstieg in Ihr Abstract, wiederholen Sie im Einstiegssatz also nicht den Titel der Arbeit ("Die vorliegende Arbeit befasst sich mit [Titel]").
Schreiben Sie das Abstract erst, wenn die gesamte Arbeit fertig ist. Die Grundlage für das Abstract bilden die Einleitung und das Schlusskapitel (Zusammenfassung). Die Inhalte dieser beiden Kapitel finden in stark geraffter Form Eingang in das Abstract.
Auch wenn es Ihnen auf den ersten Blick mühsam erscheint: Kopieren Sie nicht Sätze aus der Einleitung oder dem Schlusskapitel in Ihr Abstract, sondern schreiben Sie das Abstract neu. Nur so wird es Ihnen gelingen, einen aussagekräftigen Text zu formulieren.

}
